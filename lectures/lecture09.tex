\chapter{Feb.~11 --- Real Forms}

\section{Automorphisms of Semisimple Lie Algebras}

\begin{remark}
  Recall that we can identify
  $\Aut(\g)$ with a Lie group with
  Lie algebra $\g$. The connected
  component of the identity
  $\Aut^0(\g)$ (also known as the
  \emph{adjoint group} $G_{\mathrm{ad}}$)
  acts transitively
  on the set of Cartan subalgebras.
  If $\mathfrak{h} \subseteq \g$
  is a Cartan subalgebra, then there is
  a connected subgroup
  $H \subseteq G_{\ad}$ which
  acts as $1$ on $\mathfrak{h}$ and
  as $e^{\alpha(x)}$ on $g_\alpha$
  for $x \in \mathfrak{h}$. Then
  we have
  \[\mathfrak{h} / 2\pi i P^\vee \cong H,\]
  and $H$ is called a \emph{maximal torus}.
\end{remark}

\begin{prop}
  The normalizer $N(H)$ of $H$
  in $G_{\mathrm{ad}}$ coincides with the
  stabilizer of $\mathfrak{h}$ and
  contains $H$ as a normal subgroup
  such that $N(H) / H = W$ (the Weyl group).
\end{prop}

\begin{proof}
  Note that $\SL_2(\C)$ is simply connected,
  so for any simple root $\alpha_i$
  there is a homomorphism
  \[
    \eta_i : \SL_2(\C)
    \longrightarrow G_{\ad} = \widetilde{G} / \mathcal{Z}(\widetilde{G}).
  \]
  Define $S_i = \eta_i\left(\begin{smallmatrix}
    0 & 1 \\ -1 & 0
  \end{smallmatrix}\right)$.
  Consider $w = s_{i_1} \cdots s_{i_n}$
  and $\widetilde{w} = S_{i_1} \cdots S_{i_n}$.
  Note that $\widetilde{w}$ acts on
  $\mathfrak{h}$ acts $w$, and
  if $w = w_1 w_2$, then
  $\widetilde{w} = \widetilde{w}_1 \widetilde{w}_2 h$
  for some $h \in H$.
  To see the latter claim, note that $h$
  has the preserve the root decomposition,
  hence $h|_{\mathfrak{g}_{\alpha_j}} = \exp(b_j)$.
  Thus $h = \exp(\sum_j b_j w_j^\vee) \in H$
  (where $\langle w_j^\vee, \alpha_i \rangle = \delta_{j, i}$).

  So $\widetilde{w}$ and $H$ generate
  a subgroup $N \subseteq N(H)$ such that
  $N / H = W$. It remains to show that
  $N(H) = N$. Let $x \in N(H)$ and consider
  simple roots $\alpha_i' = x(\alpha_i)$.
  Then there exists $w \in W$ such that
  $w(\alpha_i') = \alpha_{p(i)}$ for some
  permutation $p$. Then
  $\widetilde{w} x(\alpha_i) = \alpha_{p(i)}$,
  so this is a Dynkin diagram automorphism.
  Now $\widetilde{\omega} x$ is an element
  of a group, so the fundamental
  weights are fixed. Thus
  $p = \id$.
\end{proof}

\begin{remark}
  Although $N(H) / H = W$, in general
  $N(H)$ is not a semidirect product
  of $H$ and $W$.
\end{remark}

\begin{prop}
  The map
    $\xi : \Aut(D) \ltimes G_{\mathrm{d}}
    \to \Aut(\g)$
  is an isomorphism.
\end{prop}

\begin{proof}
  We have to show that $\xi$ is surjective.
  Let $a \in \Aut(\g)$. We can say that
  $a$ preserves the Cartan subalgebra
  (if not, we can shift it by $g \in G_{\mathrm{ad}}$).
  Multiplying by $\Aut(D) N(H)$,
  we can make it act trivially
  on $\mathfrak{h}$ and
  $\g_{\alpha_i}$. Then $a = 1$, so
  $a \in \im \xi$.
\end{proof}

\section{Real Forms of Semisimple Lie Algebras}

\begin{remark}
  Recall the Serre presentation for
  $\mathfrak{g}$, i.e. generators
  $\{h_i, f_i, e_i\}$ with certain
  relations. In this setting, everything
  was defined
  over $\Q$.
\end{remark}

\pagebreak
\begin{definition}
  A semisimple Lie algebra
  is \emph{split} if it admits a
  Chevalley-Serre basis over
  base field $K$.
\end{definition}

\begin{remark}
  Let $L$ be a Galois extension of $K$ 
  ($\Char K = 0$), and
  assume that $\g_L$ is a split semisimple
  Lie algebra. We want to find
  $\g$ over $K$
  such that $\g \otimes_K L = \g_L$.
  The problem is then to find a
  classification of all such $\g$.
  Let
  $\Gamma = \Gal(L / K)$.
  Define an action of $\Gamma$ on $\g_L$ by
  \[
    g(\lambda x) = g(\lambda) g(x),
    \quad x \in \g_L, \lambda \in L,
    g \in \Gamma,
  \]
  which is twisted linear. We can
  reconstruct $\g$ as the
  invariants $\g_L^\Gamma$.

  The simplest action of this kind is
  $\rho_0(g)$, which acts on scalars
  and preserves $\{h_i, e_i, f_i\}$.
  Any twisted linear action
  takes the form $\rho(g) = \eta(g) \rho_0(g)$
  for some $\eta : \Gamma \to \Aut(\g_L)$.
  As $\rho$ is a homomorphism,
  \[
    \eta(gh) \rho_0(gh)
    = \eta(g) \rho_0(g) \eta(h) \rho_0(h),
  \]
  and upon rearranging, we have
  \[
    \eta(gh)
    = \eta(g) g(\eta(h)),
  \]
  where $g(a) = \rho_0(g) a \rho_0(g)^{-1}$
  for $a \in \Aut(\g_L)$.
  The above is called a
  \emph{$1$-cocycle condition}.

  Denote the Lie algebra associated
  to the cocycle $\eta$ by $\g_\eta$.
  When do we have $\g_{\eta_1} \cong \g_{\eta_2}$?
  This is the case when $\rho_1$ and
  $\rho_2$ are isomorphic, i.e. there
  exists $a \in \Aut(\g_L)$ such that
  $\rho_1(g) a = a \rho_2(g)$. Then
  \[
    \eta_1(g)
    \rho_0(g) a = a \eta_2(g) \rho_0(g),
  \]
  so $\eta_1(g) = a \eta_2(g) g(a)^{-1}$.
  Thus $\eta_1$ and $\eta_2$ are
  cohomologous cocycles.
\end{remark}

\begin{prop}
  The semisimple Lie algebras $\g$
  over $K$ which split over $L$ (where
  $L / K$ is Galois) are classified by
  $H^1(\Gamma, \Aut(\g_L))$, where
  $\Gamma = \Gal(L / K)$.
\end{prop}

\begin{remark}
  We will now specialize to
  $K = \R$, $L = \C$, where
  $\Gamma = \Z / 2\Z$, generated by
  complex conjugation.
  We have $\Aut(\g_L) = \Aut(D) \ltimes G_{\mathrm{ad}}$.
  Since $\eta(1) = 1$, $\eta$ is determined
  by $\eta(-1)$.
  The cocycle condition is
  \[
    s \overline{s} = 1, \quad s = \eta(-1).
  \]
  The corresponding Lie algebra (up to
  isomorphism) depends only on the
  cohomology class of $s$, where
  $s \to a s \overline{a}^{-1}$ for $a \in \Aut(D)$.
\end{remark}

\begin{theorem}
  The real semisimple Lie algebras
  whose complexification is $\g$ (i.e.
  the real forms of $\g$) are classified
  by $s \in \Aut(D) \ltimes G_{\mathrm{ad}}$
  such that $s \overline{s} = 1$,
  modulo the equivalence
  $s \to a s \overline{a}^{-1}$ for $a \in \Aut(D)$.
\end{theorem}

\begin{remark}
  Note that complex conjugation
  acts trivially on $\Aut(D)$.
\end{remark}

\begin{remark}
  Denote by $\g_{(s)} = \{x \in \g : \overline{x} = s(x)\}$ the
  real form corresponding to $s$.
  Denote by $\g_{(1)}$ the split
  form consisting of real $x \in \g$
  (so that $x = \overline{x}$).

  Alternatively, we can define an
  antilinear involution
  $\sigma_s(x) = \overline{s(x)}$.
  Then $\g_{(s)}$ is the fixed point
  set of $\sigma_s$.
\end{remark}

\begin{remark}
  Note that $s$ defines $s_0 \in \Aut(D)$
  with $s_0^2 = 1$.
\end{remark}

\begin{corollary}
  The conjugacy class of $s_0$ is
  invariant under equivalences.
\end{corollary}

\begin{remark}
  Since $s_0$ permutes the
  connected components of the Dynkin
  diagram $D$, it preserves some and
  divides some into pairs.
  So every semisimple real
  Lie algebra is a direct sum of simple
  ones, and each simple one has
  either connected Dynkin diagram or
  consists of two identical components.
\end{remark}

\begin{remark}
  We now consider the case when
  $D$ is connected and $\g$ is simple.
\end{remark}

\begin{definition}
  A real form $\g_{(s)}$ of a complex
  simple Lie algebra is said to be
  \emph{inner to}  $\g_{(s')}$ 
  if $s' = gs$ up to equivalence, where
  $g \in G_{\mathrm{ad}}$ (i.e. $s, s'$ differ
  by an inner automorphism).
  The \emph{inner class} of $\g_{(s)}$ is the collection
  of all real forms inner to
  $\g_{(s)}$. An \emph{inner form}
  is a form inner to a split form.
  We call $\g_{(s)}$ \emph{quasi-split}
  if $s = s_0 \in \Aut(D)$.
\end{definition}

\begin{corollary}
  We have the following:
  \begin{enumerate}
    \item Any real form is inner to a unique
      quasi-split form.
    \item A real form which is both
      inner and quasi-split is split.
  \end{enumerate}
\end{corollary}

\begin{example}
  Consider the \emph{Cartan involution}
  $\tau$ defined by
  \[
    \tau(h_j) = -h_j, \quad
    \tau(e_j) = -f_j, \quad
    \tau(f_j) = -e_j.
  \]
  Then $\g_{(\tau)} = \g^c$
  is called the \emph{compact real form}
  of $\g$.
\end{example}
