\chapter{Feb.~23 --- Classifications of Lie Groups}

\section{Classification of Compact Lie Groups}

\begin{remark}
  Let $\g$ be semisimple and $G$ the
  corresponding simply connected
  Lie group (which is the universal
  cover for $G_{\mathrm{ad}} = \Aut(\g)^0$).
  Define
  \[
    Z = \ker(G \to G_{\mathrm{ad}})
    = \pi_1(G_{\mathrm{ad}}).
  \]
  Recall that the finite-dimensional
  representations of $G$ are in one-to-one
  correponsdence with the finite-dimensional
  representations of $\g$, which
  (the irreducible ones) are given by
  $\{L_\lambda\}$ for $\lambda \in P_+$.
  Then $Z$ acts as $\chi_\lambda : Z \to \C^\times$
  on every $L_\lambda$, and
  $\chi_\lambda \chi_\mu = \chi_{\lambda + \mu}$.
  Note that
  \[
    \chi : P
    \longrightarrow \Hom(Z, \C^\times),
  \]
  and $\chi_\theta = 1$ (where $\theta$ is the
  longest root)
  as $Z$ acts trivially on the
  adjoint representation.
\end{remark}

\begin{exercise}
  If $\lambda(h_i)$ is sufficiently
  large, then show that
  for all $\alpha \in \g$ we have
  \[L_{\lambda + \alpha} \subseteq L_{\lambda} \otimes \g,\]
  so in particular,
  $\chi_{\lambda + \alpha} = \chi_\lambda$,
  i.e. $\chi_\alpha = 1$.
\end{exercise}

\begin{remark}
  We can define maps
  $P / Q \to \Hom(Z, \C^\times)$ and
  $Z \to P^\vee / Q^\vee$ (and similarly
  for $G^c_{\mathrm{ad}}, G^c, Z^c$).
\end{remark}

\begin{prop}
  A representation $L_\lambda$ of $\g$
  of highest weight $\lambda \in P_+$
  lifts to a representation of
  $G_{\mathrm{ad}}$ (equivalently, of
  $G_{\mathrm{ad}}^c$)
  if and only if $\lambda \in P_+ \cap Q$.
\end{prop}

\begin{proof}
  We have shown if
  $\lambda \in P_+ \cap Q$, then
  $L_\lambda$ lifts. The converse
  follows from
  Proposition
  \ref{prop:faithful}.
\end{proof}

\begin{prop}\label{prop:faithful}
  If $V$ is a faithful finite-dimensional
  representation of a compact Lie group,
  then
  any irrep $Y$ is contained in
  $V^{\otimes n} \otimes (V^*)^{\otimes m}$
  for some $n, m$.
\end{prop}

\begin{lemma}\label{lem:fg}
  If $X$ is a compact manifold, then
  $\pi_1(X)$ is finitely generated.
\end{lemma}

\begin{proof}
  Cover $X$ with balls around each point,
  by the compactness of $X$
  there is a finite subcover.
  Let $x_1, \dots, x_n$ be the centers of
  the finitely many balls, and create
  a graph $G$ by connecting
  $x_i$. Then there is a surjection
  $\pi_1(G) \to \pi_1(X)$, and 
  $\pi_1(G)$ is finitely generated.
\end{proof}

\begin{theorem}
  Let $\g$ be a semisimple complex
  Lie algebra and $G^c_{\mathrm{ad}}$
  the corresponding adjoint compact group.
  Then $\pi_1(G^c_{\mathrm{ad}}) = P^\vee / Q^\vee$.
  In particular, the universal cover of
  $G^c_{\mathrm{ad}}$ is a
  compact Lie group.
\end{theorem}

\begin{proof}
  Let $G_*^c$ be a finite cover of
  $G_{\mathrm{ad}}^c$ and define
  \[Z_{G_{\mathrm{ad}}^c} = \ker(G_*^c \to G_{\mathrm{ad}}^c) \subseteq G_*^c.\]
  A finite-dimensional irrep is
  classified by
  $P_+(G_*^c) \subseteq P_+$ with
  $P_+ \cap Q \subseteq P_+(G_*^c)$.
  Let $P(G_*^c) \subseteq P$ be the
  lattice generated by $P_+(G_*^c)$, and
  consider the character $\chi_\lambda$
  for the action of
  $Z_{G_{\mathrm{ad}}^c}$ on
  $L_\lambda$ (an irrep of $G_*^c$).
  Then $\chi$ gives a map
  \[
    \xi : P(G_*^c) / Q
    \longrightarrow
    Z_{G_{\mathrm{ad}}^c}^\vee
    = \Hom(Z_{G_{\mathrm{ad}}^c}, \C^\times).
  \]
  Since $G_*^c$ is compact,
  by the Peter-Weyl theorem $\xi$ is
  surjective. It just remains to show
  that $\pi_1(G^c_{\mathrm{ad}})$ is
  finite. Let
  $G_*^c = G^c$, the universal cover. Then
  $P(G_*^c) = P$, so
  $P / Q \cong Z^\vee$, and thus
  \[
    Z = \pi_1(G^c_{\mathrm{ad}}) \cong P^\vee / Q^\vee.
  \]
  By Lemma \ref{lem:fg}, $\pi_1(G^c_{\mathrm{ad}})$ is finitely generated, and it
  is also abelian. Take a subgroup of
  finite index $N$ and take
  $G_*^c$ to be the corresponding cover.
  Then we have
  \[
    N = |Z_{G_*^c}|
    \le |P(G_*^c) / Q|
    \le |P / Q|,
  \]
  which for abelian groups implies that
  the group is finite.
\end{proof}

\begin{corollary}
  We have the following:
  \begin{enumerate}
    \item If $\g$ is a semisimple
      complex Lie algebra, then the simply
      connected Lie group $G^c$
      corresponding to the Lie algebra $\g^c$
      is compact and its center
      is $P^\vee / Q^\vee$, which is
      the same as $\pi_1(G^c_{\mathrm{ad}})$.
    \item Let $\Gamma \subseteq P^\vee / Q^\vee$.
      Then the irreps of $G / \Gamma$
      are the $L_\lambda$ such that
      $\lambda$ defines the trivial
      character of $\Gamma$.
    \item Let $G_i^c$ be compact Lie groups
      corresponding to simple
      Lie algebras $\g_i$, and let
      $\g = \bigoplus_{i = 1}^n \g_i$. Then
      any connected Lie group with Lie
      algebra $\g^c$ is compact and
      of the form
      \[
        \bigg(\prod_{i = 1}^n G_i^c\bigg) / Z,
      \]
      where $Z = \pi_1(G^c)$
      is a subgroup of $\prod_i Z_i$ for
      $Z_i = P_i^\vee / Q_i^\vee$ (the
      center of $G_i^c$). Moreover,
      every semisimple connected compact
      Lie group has this form.
  \end{enumerate}
\end{corollary}

\begin{example}
  Let $G_*^c = \SO(4, \R)$. Then
  $G_{\mathrm{ad}}^c = \SO(3, \R) \times \SO(3, \R)$ and
  $G^c = \SU(2) \times \SU(2)$, and
  \[
    \SO(4, \R)
    = (\SU(2) \times \SU(2)) / \{\pm (1, 1)\}.
  \]
\end{example}

\begin{example}
  Let $G_* = \SO(4, \C)$. Then
  $G = \SL(2, \C) \times \SL(2, \C) = \mathrm{Spin}(4)$,
  and
  \[
    \SO(4, \C)
    = (\SL(2, \C) \times \SL(2, \C)) / \{\pm (1, 1)\}.
  \]
  In this case, $G_{\mathrm{ad}} = \PSL(2, \C) \times \PSL(2, \C)$.
\end{example}

\begin{corollary}
  Any connected compact Lie group with
  Lie algebra of the form
  $\g^c \oplus \mathfrak{a}$
  with $\mathfrak{a}$ abelian
  is a quotient of
  $T \times C$ by a finite central subgroup,
  where $T = (S^1)^m$ and $C$ is
  compact, semisimple, and simply connected.
\end{corollary}

\section{Polar Decomposition}
\begin{remark}
  Consider $G_{\mathrm{ad}, \theta} \subseteq G_{\mathrm{ad}}$
  corresponding to $\g_\theta \subseteq \g$.
  Note that $G_{\mathrm{ad}, \theta}$
  is a closed subgroup, but it may be
  disconnected (e.g. if $\g_\theta = \mathfrak{sl}(2, \R) \subseteq \mathfrak{sl}(2, \C)$, then
  $G_{\mathrm{ad}} = \mathrm{PGL}_2(\C)$ and
  $G_{\mathrm{ad}, \theta} = \mathrm{PGL}_2(\R)$, which is disconnected as
  $\det : \GL_2(\R) \to \R \setminus \{0\}$
  and $\R \setminus \{0\}$ is disconnected;
  but $\C \setminus \{0\}$ is connected).

  \pagebreak

  Now let $K^c \subseteq G_{\mathrm{ad}, \theta}$
  be the subgroup of elements acting
  on $\g$ by unitary operators, i.e.
  $K^c$ is the fixed points of
  $\omega_\theta$. Note that $K^c$ is a
  closed but possibly disconnected subgroup.
  Let $\Lie(K^c) = \mathfrak{k}^c$, and
  note that $K^c$ is compact.
  Let $P_\theta = \exp(\mathfrak{p}_\theta) \subseteq G_{\mathrm{ad}, \theta}$,
  where $\mathfrak{p}_\theta = i\mathfrak{p}^c$ (note that
  $P_\theta$ need not be a subgroup).
  Now $P_\theta$ acts on $\g$ by Hermitian
  operators, so we get a diffeomorphism
  \[
    \exp : \mathfrak{p}_\theta \to P_\theta,
  \]
  thus $P_\theta$ is a closed embedded
  submanifold in $G_{\mathrm{ad}, \theta}$.
\end{remark}

\begin{theorem}[Polar decomposition]
  The multiplication
  $K^c \times P_\theta \to G_{\mathrm{ad}, \theta}$ is a diffeomorphism,
  hence
  \[
    G_{\mathrm{ad}, \theta}
    \cong K^c \times \R^{\dim \mathfrak{p}}
  \]
  as manifolds. In particular,
  $G_{\mathrm{ad}, \theta}$ is
  homotopy equivalent to $K^c$.
\end{theorem}

\begin{proof}
  By the polar decomposition for
  matrices,
  any invertible matrix $A$ can be written
  as $A = U_A R_A$, where
  $U_A$ is unitary and $R_A$ is
  positive Hermitian. Explicitly, we have
  \[
    R_A = (A^\dagger A)^{1/2}
    \quad \text{and} \quad
    U_A = A (A^\dagger A)^{-1/2}.
  \]
  Let $g \in G_{\mathrm{ad}, \theta} \subseteq \Aut(\g) \subseteq \GL(\g)$.
  Note that $g^\dagger g$ is an
  automorphism of $\g$ with positive
  eigenvalues, so
  $(g^\dagger g)^{1 / 2} = R_g \in P_\theta$,
  the positive self-adjoint elements.
  Since $U_g$ is unitary, it has to
  belong to $K^c$. Thus we have constructed
  an inverse map to $\mu$.
\end{proof}

\begin{corollary}
  We have $G_{\mathrm{ad}} \cong G^{\mathrm{ad}} \times \mathbb{P}$,
  where $\mathbb{P}$ is the set of elements
  acting on $\g$ by positive Hermitian
  operators. In particular,
  $\pi_1(G_{\mathrm{ad}}) = \pi_1(G_{\mathrm{ad}}^c) = P^\vee / Q^\vee$.
\end{corollary}

\begin{corollary}
  If $G$ is a semisimple complex Lie group,
  then the center $Z$ of $G$ is contained
  in $G^c$,
  hence $Z$ coincides with the center
  $Z^c$ of $G^c$. Thus the restriction
  of finite-dimensional representations
  from $G$ to $G^c$ defines an
  equivalence of categories.
\end{corollary}

\begin{remark}
  By considering coverings, the polar
  decomposition also applies to the
  real form $G_\theta \subseteq G$ of any
  connected complex semisimple Lie
  group $G$. However, note that
  $G_\theta$ need not be simply
  connected even if $G$ is (e.g.
  for $G = \SL_2(\C)$ we have
  $G_\theta = \SL_2(\R) \cong \SO(2) \cong S^1$,
  so $\pi_1(G_\theta) \cong \Z$).
\end{remark}

\begin{example}
  We have the following:
  \begin{enumerate}
    \item If $G = \SL_n(\C)$, then
      $K^c = \SU(n)$ and $P$
      is the set of positive Hermitian
      matrices of determinant $1$.
    \item If $G_\theta = \SL_n(\R)$,
      then $K^c = \SO(n)$ and
      $P_\theta$ is the positive
      symmetric matrices of determinant
      $1$.
  \end{enumerate}
\end{example}
