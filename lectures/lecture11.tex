\chapter{Feb.~18 --- Real Forms, Part 3}

\section{Classification of Real Forms, Continued}

\begin{prop}
  There exists a Cartan subalgebra
  $\mathfrak{h}$ in $\g$ which is invariant
  under $\theta$ and such that
  $\mathfrak{h} \cap \mathfrak{k}$
  is a Cartan subalgebra in $\mathfrak{k}$.
\end{prop}

\begin{proof}
  Consider a generic element
  $t \in \mathfrak{k}^c$. It is regular and
  semisimple. Consider
  $\mathfrak{h}^c_+$, the centralizer
  of $t$ in $\mathfrak{k}^c$. Then
  necessarily $\mathfrak{h}_+ := \mathfrak{h}^c_+ \otimes_\R \C$
  is a Cartan subalgebra in $\mathfrak{k}$.
  Let $\mathfrak{h}^c_-$ be a maximal
  subspace of $\mathfrak{p}^c$ so that
  $\mathfrak{h}^c = \mathfrak{h}^c_+ \oplus \mathfrak{h}_-^c$
  is a commutative subalgebra of
  $\g^c$.
  Then we claim that $\mathfrak{h} = \mathfrak{h}^c \otimes_\R \C$
  is a Cartan subalgebra of $\g$.
  Note that $\mathfrak{h}$ consists of
  the semisimple elements, and all
  elements in $\g^c$ are
  anti-self-adjoint operators.
  If $z \in \g$ commutes with
  $\mathfrak{h}$, then
  \[
    z = z_+ + z_-, \quad z_+ \in \mathfrak{k}, z_- \in \mathfrak{p},
  \]
  where $z_\pm$ commute with $\mathfrak{h}$.
  Then $z_+ \in \mathfrak{h}_+$, and
  \[
    z_- = x + iy, \quad x, y \in \mathfrak{p}^c,
  \]
  where $x, y$ commute with
  $\mathfrak{h}$. Then
  $x, y \in \mathfrak{h}^c_-$ (by the
  definition of $\mathfrak{h}^c_-$),
  so $z \in \mathfrak{h}$.
\end{proof}

\begin{corollary}
  We have
  $\mathfrak{h} = \mathfrak{h}_+ \oplus \mathfrak{h}_-$, where
  $\theta = 1$ on $\mathfrak{h}_+$ and
  $\theta = -1$ on $\mathfrak{h}_-$.
\end{corollary}

\begin{lemma}
  There are no coroots of $\g$ in
  $\mathfrak{h}_-$.
\end{lemma}

\begin{proof}
  Suppose otherwise
  that $\alpha^\vee \in \mathfrak{h}_-$.
  Then $\theta(\alpha^\vee) = -\alpha^\vee$,
  so
  \[
    \theta(e_\alpha) = e_{-\alpha}
    \quad \text{and} \quad
    \theta(e_{-\alpha}) = e_\alpha
  \]
  for some $e_{\pm \alpha} \in \g_{\pm \alpha}$.
  Then $x = e_{\alpha} + e_{-\alpha}$
  satisfies $\theta(x) = x$, so
  $x \in \mathfrak{k}$. But
  $x \notin \mathfrak{h}_+$ (since
  $x \perp \mathfrak{h}_+$).
  Thus $[\mathfrak{h}_+, x] = 0$
  since $\alpha$ vanishes on
  $\mathfrak{h}_+$, a contradiction as
  $\mathfrak{h}_+$ is a maximal commutative
  subalgebra of $\mathfrak{k}$.
\end{proof}

\begin{remark}
  Pick a generic element $t \in \mathfrak{h}_+$,
  which is regular in $\g$. Choose $t$
  so that
  \[
    \re(t, \alpha^\vee)
    \ne 0
  \]
  for all $\alpha^\vee$ of $\g$. Then
  we can define a \emph{polarization}
  on $R$ by
  \[
    R_+ = \{\alpha \in R : \re(t, \alpha^\vee) > 0\}
  \]
  which satisfies $\theta(R_+) = R_+$.
  Now $\{\theta(i) : i \in D\}$ gives the
  action of $\theta$: If $\theta = i$, then
  \[
    \theta(e_i) = \pm e_i, \quad
    \theta(h_i) = h_i, \quad
    \theta(f_i) = \pm f_i.
  \]
  \pagebreak
  Otherwise, if $\theta(i) \ne i$,
  then we can choose generators
  $h_i, e_i, e_{\theta(i)}, f_i, f_{\theta(i)}, h_{\theta(i)}$
  such that
  \[
    \theta(x_i) = x_{\theta(i)}, \quad
    x = e, f, h.
  \]
  We can then construct \emph{markings}
  on the Dynkin diagram as follows:
  \begin{itemize}
    \item Connect vertices $i$ and
      $\theta(i)$ if $\theta(i) \ne i$.
    \item Mark a vertex $i$ as white if
      $\theta(e_i) = e_i$.
    \item Mark a vertex $i$ as black if
      $\theta(e_i) = -e_i$.
  \end{itemize}
  This is the \emph{Vogan diagram}
  associated to the Dynkin diagram.
  Note that $e_i \in P$ is a non-compact root.
\end{remark}

\begin{exercise}
  Showing the following:
  \begin{enumerate}
    \item The signature of the Killing
      form of $g_\theta$ is
      $(\dim \mathfrak{p}, \dim \mathfrak{k})$.
      Moreover, the Killing form is
      negative definite if and only if
      $\theta = 1$, i.e. $\g = \g^c$.
    \item For a split real form,
      $\dim \mathfrak{k} = |R_+|$.
    \item Show that for any real form
      in a compact inner class,
      $\rank(\mathfrak{k}) = \rank(\g)$.
  \end{enumerate}
\end{exercise}

\section{Real Forms of Classical Lie Algebras}

\begin{example}
  We have the following real forms of
  the classical Lie algebras:
  \begin{enumerate}
    \item Type $A_{n - 1}$, compact inner
      class.

      Let $\theta$ be the inner automorphism
      element of $\mathrm{PSU}(n)$
      of order $2$. Let
      $g \in \U(n)$ such that $g^2 = 1$.
      Then $\theta(x) = gxg^{-1}$, so
      $g = \id_p \otimes (-\id_q)$
      with $p + q = n$. Thus
      \[
        \g_\theta = \mathfrak{su}(p, q)
        \quad \text{and} \quad
        \mathfrak{k}
        = \mathfrak{gl}_p \oplus \mathfrak{sl}_q.
      \]
      For $n = 2$, we have
      $\mathfrak{su}(2)$ and
      $\mathfrak{su}(1, 1)$, with
      $\mathfrak{k} = \mathfrak{gl}_1$.
    \item Type $A_{n - 1}$, split inner class.

      If $n$ is odd (so all vertices
      are divided into connected pairs),
      then
      \[
        \g_\theta = \mathfrak{sl}_n(\R).
      \]
      If $n$ is even, then there is
      $1$ stable vertex (which is either
      black or white). In these cases
      we either have
      $\mathfrak{k} = \mathfrak{sp}_{2k}$
      (in which case $\g_\theta = \mathfrak{sl}(k, \mathbb{H})$) or
      $\mathfrak{k} = \mathfrak{so}_{2k}$
      (which is just the split form
      $\mathfrak{sl}_n(\R)$).
    \item Type $B_n$ (i.e. $\mathfrak{so}_{2n + 1}$).

      Let $\theta$ be the inner automorphism
      of order $\le 2$. We can write
      \[
        \theta = \id_{2p + 1} \oplus (-\id_{2q})
      \]
      where $p + q = n$. The real forms
      are $\mathfrak{so}(2p + 1, 2q)$,
      with $\mathfrak{k} = \mathfrak{so}_{2p + 1} \oplus \mathfrak{so}_{2q}$.
    \item Type $C_n$.

      We can have
      $g \in \Sp_{2n}(\C)$ such that
      $g^2 = 1$ or $g^2 = -1$. The adjoint
      compact group is
      \[(\Sp(2n) \cap U(n)) / {\pm 1}.\]
      If $g^2 = 1$, then the eigenspace
      with eigenvalue $1$ has dimension
      $2p$, and the eigenspace for
      $-1$ has dimension $2q$ (where
      $p + q = n$). Assume that
      $p \ge q$ (otherwise take $g \mapsto -g$).
      Then we have
      \[
        \mathfrak{sp}(2p, 2q)
        = \mathfrak{sp}_{2n} \cap \mathfrak{u}(p, q)
        = \mathfrak{u}(p, q, \mathbb{H}).
      \]
      In this case, we have
      $\mathfrak{k} = \mathfrak{sp}_{2p} \oplus \mathfrak{sp}_{2q}$.

      If $g^2 = -1$, then
      $\C^{2n} = V(i) \oplus V(-i)$.
      In this case, $\mathfrak{k} = \mathfrak{gl}_n(\C)$,
      as for any $(w, w) = w \cdot \overline{w}$
      in $\C^n$,
      \[
        \im(w, w)
        = i \im (w \wedge \overline{w})
      \]
      defines a symplectic form
      on $\R^{2n}$. Thus we can view
      $\U(n) \subseteq \Sp_{2n}(\R)$.

    \item Type $D_n$, compact inner class.

      In this case, $\theta$ is given by
      $g \in \SO(2n)$ with $g^2 = \pm 1$.

      If $g^2 = 1$, then
      $\C^{2n} = V(1) \oplus V(-1)$ (
      where $\dim V(1) = 2p$ and $\dim V(-1) = 2q$ with $p + q = n$).
      Note that $\det(g) = 1$, so the
      eigenspaces are even-dimensional.
      Then
      \[
        \g_\theta
        = \mathfrak{so}(2p, 2q)
        \quad \text{and} \quad
        \mathfrak{k}
        = \mathfrak{so}(2p) \oplus \mathfrak{so}(2q).
      \]
      If $g^2 = -1$, then
      $\C^{2n} = V(i) \oplus V(-i)$,
      so $\mathfrak{k} = \mathfrak{gl}_n(\C)$.
      In this case, we have
      \[
        \g_\theta = \mathfrak{so}^*(2n),
      \]
      which is known as the
      \emph{quaternionic orthogonal Lie algebra}.
    \item Type $D_n$, the other inner class.

      In this case, $\theta$ is given by
      $g \in \OO(2n)$ with
      $\det(g) = -1$ and $g^2 = \pm 1$.

      Note that if $g^2 = -1$, then $\det(g) = 1$,
      which is a contradiction. So
      we can only have $g^2 = 1$. Then
      \[
        \C^{2n} = V(1) \oplus V(-1),
      \]
      where $\dim V(1) = 2p + 1$ and $\dim V(-1) = 2q + 1$.
      Here $\mathfrak{k} = \mathfrak{so}({2p + 1}) \oplus \mathfrak{so}({2q + 1})$.
  \end{enumerate}
\end{example}

\section{More on Compact Groups}

\begin{exercise}
  Show that if $K^c$ is a
  compact Lie group, then
  $\mathfrak{k} = \Lie_\C(K^c)$ is a
  reductive Lie algebra.
\end{exercise}

\begin{example}
  Let $G_{\mathrm{ad}} = \Aut^0(\g)$
  for a semisimple Lie algebra $\g$,
  and let $G_{\mathrm{ad}}^c$ be
  its compact form. Consider the
  following product:
  \[
    (S^1)^r \times G_{\mathrm{ad}}^c.
  \]
  We will see that any Lie algebra
  of a compact group is isomorphic to a
  Lie algebra of such a product.
  We can also consider covering spaces,
  i.e. what is $\pi_1(G_{\mathrm{ad}}^c)$?
\end{example}
