\chapter{Jan.~14 --- Applications of Schur-Weyl Duality}

\section{The Schur Functor}

\begin{remark}
  Let $V$ be the defining
  representation for $\GL_n$. Then
  \[
    V^{\otimes N}
    = \bigoplus_{\lambda : |\lambda| = N} L_\lambda \otimes \pi_\lambda.
  \]
  Recall that if
  $\lambda = (\lambda_1, \dots, \lambda_n)$, then
  we have
  \[
    \lambda_1 = m_1 + \dots + m_n,
    \quad \lambda_2 = m_2 + \dots + m_n, \quad \dots, \quad
    \lambda_n = m_n.
  \]
\end{remark}

\begin{definition}
  Suppose we are given the
  partition $\lambda$ of $N$.
  The \emph{Schur functor} $S^\lambda$
  is given by
  \[
    S^\lambda V
    = \Hom_{S_N}(\pi_\lambda, V^{\otimes N})
  \]
  for a vector space $V$.
  Note that this language, we have
  $V^{\otimes N} = \bigoplus_\lambda S^\lambda V \otimes \pi_\lambda$.
\end{definition}

\begin{example}
  Consider the following:
  \begin{enumerate}
    \item $S^{(n)} V = S^n V$, where
      $(n)$ is the partition
      of $n$ with a single part.
    \item $S^{(1^n)} V = \wedge^n V$, where
      $(1^n)$ is the partition of $n$
      with $n$ parts equal to $1$.
    \item $V \otimes V = S^{(2)} V \otimes \C_+ \oplus S^{(1, 1)} V \otimes \C_-$,
      where $\C_2$ acts trivially on
      $\C_+$ and
      by the sign on $\C_-$.
    \item $V \otimes V \otimes V = S^{(3)} V \otimes \C_+ \oplus S^{(2, 1)} V \otimes \C^2 \oplus S^{(1, 1, 1)} V \otimes \C_-$,
      where $S_3$ acts trivially
      on $\C_+$ and by sign on
      $\C_-$ as before, and
      $\C^2 = \{(x, y, z) : x + y + z = 0\}$.

      Note that $V \otimes V = S^2 V \oplus \wedge^2 V$, so
      $S^2 V \otimes V = S^3 V \oplus S^{(2, 1)} V$ and
      $\wedge^2 V \otimes V = \wedge^3 V \oplus S^{(2, 1)} V$.
  \end{enumerate}
\end{example}

\begin{remark}
  Let $\dim V = N$ and $\lambda$ have
  $k$ parts.
  Recall that by the Weyl dimension
  formula,
  \[
    \dim L_\lambda
    = \frac{\prod_{\alpha \in R_+} (\alpha, \lambda + \rho)}{\prod_{\alpha \in R_+} (\alpha, \rho)}.
  \]
  We have
  $R_+ = \{\alpha_{i, j} = e_i - e_j : i < j\}$
  and $\rho = \sum_{i = 1}^{N - 1} \omega_i = (N - 1, N - 2, \dots, 1, 0)$
  (recall that $\omega_i$ is $i$
  ones followed by zeros). Thus we
  see that
  \[
    \dim S^\lambda V
    = \prod_{1 \le i < j \le N} \frac{\lambda_i - \lambda_j + j - i}{j - i}
    = \prod_{1 \le i < j \le k} \frac{\lambda_i - \lambda_j + j - i}{j - i}
    \prod_{1 \le i < k < j \le N}
    \frac{\lambda_i + j - i}{j - i}.
  \]
  We can rewrite the second product as
  \[
    \prod_{1 \le < k < j \le N}
    \frac{\lambda_i + j - i}{j - i}
    = \prod_{i = 1}^k
    \frac{(N + 1 - i) \cdots (N + \lambda_i - i)}{(k + 1 - i) \cdots (k + \lambda_i - i)}.
  \]
\end{remark}

\begin{prop}
  We have $\dim S^\lambda V = P_\lambda(N)$,
  where $P_\lambda$ is a polynomial
  of degree $|\lambda|$ with rational
  coefficients and integer roots.
  The roots of $P_\lambda$ are all
  integers from the interval
  $[1 - \lambda_1, k - 1]$
  (occurring with multiplicities).
\end{prop}

\begin{example}
  Let $P_n(N)$ correspond to $S^n V$.
  Then $\lambda_1 = n$ and $k = 1$, and
  \[
    P_n(N)
    = \dim S^n V
    = \binom{N + n - 1}{n}.
  \]
  Similarly, one can see that
  \[
    P_{1^n}(N)
    = \dim \wedge^n V = \binom{N}{n}.
  \]
  One can also consider
  $P_{(a, b)}(N)$ corresponding
  to partitions with
  two parts.
  The values $P_{(a, n)}(N)$
  are called the
  Narayana numbers, which are of use
  in combinatorics.
\end{example}

\section{Invariant Theory}

\begin{remark}
  Let $V$ be a finite-dimensional
  vector space and
  $\{T_i\} \in (V^*)^{\otimes m_i} \otimes V^{\otimes n_i}$
  for $i = 1, \dots, k$. One would
  like to characterize \emph{invariants}
  of such collections, i.e.
  polynomial functions
  $F(T_1, \dots, T_k)$ which are
  invariant under the action of
  $\GL(V)$.

  One can think of such a tensor
  in $(V^*)^{\otimes m_i} \otimes V^{\otimes n_i}$
  as a vertex with $m_i$ incoming
  edges and $n_i$ outgoing edges.
  Then constructing invariants
  $\{T_i\}$ reduces to studying
  graphs where $T_i$ corresponds to
  a vertex $v_i$ of the graph $\Gamma$.
  This allows us to assign to a given
  graph an invariant function
  $F_{\Gamma}$.
\end{remark}

\begin{theorem}
  The functions $F_\Gamma$
  for various $\Gamma$ span the
  space of invariant functions.
\end{theorem}

\begin{proof}
  We can view an invariant as an
  invariant element of the space
  $\bigotimes_{i = 1}^k ((V^*)^{\otimes m_i} \otimes  V^{\otimes n_i})$,
  which we can view as
  $\End_{\GL(n)}(V^{\otimes M}, V^{\otimes N})$,
  where $M = \sum d_i m_i$ (the number
  of incoming edges) and
  $N = \sum d_i n_i$ (the number of
  outgoing edges).
  Note that this space is empty
  when $M \ne N$, and the
  statement follows by Schur-Weyl
  duality when $M = N$.
\end{proof}

\begin{example}
  Let $m_i = n_i = 1$. Then
  $T_1, \dots, T_k$ are matrices.
  Then the graph $\Gamma$ must look
  like a cycle, hence the invariants
  are all of the form
  \[
    F_{j_1, \dots, j_r}
    (T_1, \dots, T_k)
    = \tr(T_{j_1} \cdots T_{j_r}).
  \]
  Note that these invariants
  are asymptotically algebraically
  independent (when $V$ is large enough).
  In particular,
  if $P(T_1, \dots, T_k) = 0$ in
  all dimensions, then
  $\tr(P(T_1, \dots, T_k) T_{k + 1}) = 0$,
  which cannot be true as the trace
  decomposes in terms of
  the $F_{j_1, \dots, j_r}$.
  (However, note that
  $[X, Y] = 0$ for $1 \times 1$
  matrices and
  $[Z, [X, Y]^2] = 0$ for
  $2 \times 2$ matrices.)
  This also implies the uniqueness
  of the $\mu_n$ in the BCH formula:
  \[
    \log (\exp(x) \exp(y))
    = \sum_{n \ge 1} \frac{\mu_n(x, y)}{n!}.
  \]
\end{example}

\section{Weyl Character Formula for \texorpdfstring{$\GL_n$}{GLn}}

\begin{remark}[Weyl character formula for $\GL_n$]
  Recall that Weyl's character formula
  gives
  \[
    \chi_{\lambda}
    = \frac{\sum_{w \in W} (-1)^{\ell(w)} e^{w(\lambda + \rho)}}
    {\sum_{\alpha \in R_+} (e^{\alpha / 2} - e^{-\alpha / 2})}, \tag{$*$}
  \]
  where the denominator
  is $\Delta = \prod_{\alpha \in R_+} (e^{\alpha / 2} - e^{-\alpha / 2}) = \prod_{w \in W} (-1)^{\ell(w)} e^{w(\rho)}$.
  Letting $\rho = \frac{1}{2} \sum_{\alpha \in R_+} \alpha$,
  \[
    \Delta = e^\rho \prod_{\alpha \in R_+} (1 - e^{-\alpha})
    = x_1^{n - 1} x_2^{n - 2} \cdots x_n^0
    \prod_{i < j} (1 - x_j / x_i),
  \]
  where $\rho = (n - 1, n - 2, \dots, 1, 0)$
  and $x_i = e^{e_i}$ (e.g.
  $x_1 = e^{(1, 0, \dots, 0)}$).
  After multiplying we get that
  \[
    \Delta = \prod_{i < j} (x_i - x_j).
  \]
  On the other hand, using
  $\Delta = \prod_{w \in W} (-1)^{\ell(w)} e^{w(\rho)}$,
  we have
  \[
    \Delta = \sum_{w \in W} (-1)^{\ell(w)} e^{w(\rho)}
    = \sum_{s \in S_n} \sign(s) x^{n - 1}_{s(1)} \cdots x^0_{s(n)}.
  \]
  Comparing these two formulas,
  we recover the formula for the
  Vandermonde determinant:
  \[
    \det(\{x_j^{n - i}\}_{1 \le i, j \le n})
    = \sum_{s \in S_n} \sign(s) x^{n - 1}_{s(1)} \cdots x^0_{s(n)}
    = \prod_{i < j} (x_i - x_j).
  \]
  Now applying this to the numerator of
  $(*)$, we have
  \[
    \sum_{w \in W} (-1)^{\ell(w)} e^{w(\lambda + \rho)}
    = \sum_{s \in S_n} \sign(s) x^{\lambda_1 + n - 1}_{s(1)} \cdots x^{\lambda_n + 0}_{s(n)}.
  \]
  Thus in total, the
  character $\chi_{\lambda}$ is given by
  \[
    \chi_{\lambda}
    = \frac{\sum_{s \in S_n} \sign(s) x^{\lambda_1 + n - 1}_{s(1)} \cdots x^{\lambda_n + 0}_{s(n)}}
    {\prod_{i < j} (x_i - x_j)}
    = \frac{\det(\{x_i^{\lambda_j + n - i}\})}{\prod_{i < j} (x_i - x_j)}.
  \]
  These functions are known as the
  \emph{Schur polynomials}
  $s_\lambda(x_1, \dots, x_n)$.
\end{remark}

\begin{example}[Character of $S^{(n)} V$]
  Using the above formula, we get the identity
  \[
    s_{(m)}(x_1, \dots, x_n)
    = \sum_{1 \le j_1 \le \cdots \le j_m \le n} x_{j_1} \cdots x_{j_m}
    = h_m(x_1, \dots, x_m),
  \]
  the $m$th complete symmetric function.
\end{example}

\begin{example}[Character of $\lambda^n V$]
  Similarly, one gets the identity
  \[
    s_{(1^m)}(x_1, \dots, x_n)
    = \sum_{1 \le j_1 < \cdots < j_m \le n} x_{j_1} \cdots x_{j_m}
    = e_m(x_1, \dots, x_m),
  \]
  the $m$th elementary symmetric function.
\end{example}

\begin{example}[Trace in $V^{\otimes N}$]
  Consider $x \otimes \sigma$, where
  $x = \diag(x_1, \dots, x_n)$ and
  $\sigma$ has $m_i$ cycles of
  length $i$. Then we have
  \[
    \tr_{V^{\otimes N}}(x \otimes \sigma)
    = \prod_i (x_1^i + \dots + x_n^i)^{m_i}.
  \]
  By Schur-Weyl duality, we have that
  \[
    \tr_{V^{\otimes N}}(x \otimes \sigma)
    = \sum_{\lambda} \chi_{\lambda}(\sigma) s_\lambda(x)
    = \prod_i (x_1^i + \dots + x_n^i)^{m_i}.
  \]
  Using the formula for the Schur
  polynomial, we get the identity
  \[
    \sum_{\lambda} \chi_\lambda(\sigma)
    \det(\{x_i^{\lambda_j + N - j}\})
    = \prod_{i < j} (x_i - x_j)
    \prod_i (x_1^i + \dots + x_n^i)^{m_i}.
  \]
\end{example}

\begin{theorem}[Frobenius character formula]
  $\chi_\lambda(\sigma)$ is the
  coefficient of $x_1^{\lambda_1 + N - 1} \cdots x_N^{\lambda_N}$
  in the polynomial
  \[\prod_{i < j} (x_i - x_j) \prod_i (x_1^i + \dots + x_n^i)^{m_i}.\]
\end{theorem}

\section{Howe Duality}

\begin{remark}
  Fix $V, W$ and consider
  $S^n(V \otimes W)$, which
  is a representation of
  $\GL(V) \otimes \GL(W)$.
\end{remark}

\begin{theorem}[Howe duality]
  We have a decomposition
  \[
    S^n(V \otimes W)
    = \bigoplus_{\lambda : |\lambda| = n} S^\lambda V \otimes S^\lambda W.
  \]
\end{theorem}

\begin{proof}
  We can write
  \[
    S^n(V \otimes W)
    = ((V \otimes W)^{\otimes n})^{S_n}
    = (V^{\otimes n} \otimes W^{\otimes n})^{S_n}.
  \]
  Using Schur-Weyl duality for each
  part, we get that
  \begin{align*}
    S^n(V \otimes W)
    &= \Bigg(\Big(\bigoplus_{\lambda : |\lambda| = n} S^\lambda V \otimes \pi_\lambda\Big)
    \otimes 
    \Big(
      \bigoplus_{\mu : |\mu| = n}
      S^\mu W \otimes \pi_{\mu}
    \Big)\Bigg)^{S_n} \\
    &= \bigoplus_{\lambda, \mu : |\lambda| = |\mu| = n}
    S^{\lambda} V \otimes
    S^{\mu} W \otimes (\pi_\lambda \otimes \pi_\mu)^{S_n}.
  \end{align*}
  Since $\pi_\lambda = \pi_{\lambda}^*$,
  by Schur's lemma we have
  $(\pi_\lambda \otimes \pi_\mu)^{S_n} = \C^{\delta_{\lambda, \mu}}$.
\end{proof}

\begin{corollary}[Cauchy identity]
  Let $x = (x_1, \dots, x_r)$ and
  $y = (y_1, \dots, y_s)$. Then
  \[
    \sum_\lambda
    s_\lambda(x) s_\lambda(y)
    z^{|\lambda|}
    = \prod_{i = 1}^r \prod_{j = 1}^s
    \frac{1}{1 - z x_i y_i}.
  \]
\end{corollary}
