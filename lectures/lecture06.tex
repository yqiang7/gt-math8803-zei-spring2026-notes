\chapter{Feb.~2 --- Duals, Maximal Weights, Exponents}

\section{Dual Representations}

\begin{remark}
  Let $L_\lambda$ be the
  irreducible representation of
  highest weight $\lambda$.
  What is the highest weight of the
  dual representation $L_\lambda^*$?
  Let $w_0$ be the maximal element
  in $W$.
\end{remark}

\begin{prop}
  We have $L_\lambda^* = L_{-w_0(\lambda)}$.
\end{prop}

\begin{proof}
  Since $\lambda$ is the highest weight
  in $L_\lambda$, for every
  weight $\mu$ in $L_\lambda$ we have
  $\lambda - \mu \in Q_+$. So
  \[
    Q_- \ni
    w_0(\lambda - \mu)
    = w_0(\lambda) - w_0(\mu),
  \]
  so $w_0(\mu) - w_0(\lambda) \in Q_+$.
  Thus $w_0(\lambda) \le w_0(\mu)$
  for all $\mu \in L_\lambda$, so the
  length of $w_0$ is $|R_+|$.
  Thus $-w_0(\lambda)$ is the
  lowest weight of $L_\lambda$, which
  is the highest weight of
  $L_\lambda^*$.
\end{proof}

\begin{example}
  Since the length of $w_0$ is
  $|R_+|$,
  $w_0$ permutes the fundamental
  (co)weights and (co)roots, so
  $w_0$ is an automorphism
  of Dynkin diagrams. Note that
  that $W$ acts on $P / Q$, and
  $w_0$ acts as inversion.
  \begin{itemize}
    \item
  The Dynkin diagrams
  $A_1, B_n, C_n, G_2, F_4, E_7, E_8$
  have no automorphisms, so
  $L_\lambda^* = L_\lambda$ for these.
  \item 
  For $A_n$ with $n \ge 2$, we have
  $P / Q = \Z / n\Z$ (e.g. if
  $V$ is the defining representation,
  then we have that $L_{\omega_1}^* = V^* = \wedge^{n - 1} V = L_{\omega_{n - 1}}$).
  \item
  For $E_6$, we have
  $P / Q = \Z / 3\Z$, where
  $w_0$ exchanges the two
  minuscule weights.
\item For $D_{2n + 1}$, we have
  $P / Q = \Z / 4\Z$ and
  $S_+^* = S_-$. For
  $D_{2n}$,
  $P / Q = \Z / 2\Z \times \Z / 2\Z$,
  and $S_{\pm}^* = S_{\pm}$.
  \end{itemize}
\end{example}

\section{Maximal Weights}

\begin{definition}
  Let \emph{maximal weight} of
  $\mathfrak{g}$, denoted $\theta$, is
  the highest weight of the
  adjoint representation.
\end{definition}

\begin{example}
  If $\g = \mathfrak{sl}_n$,
  then $\theta$ is the highest
  weight for $V^* \otimes V$
  where $V$ is the defining representation.
  Note that $V^* = \wedge^{n - 1} V$,
  so the highest weight of
  $V^* \otimes V$ is
  $\theta = \omega_1 + \omega_{n - 1}$.
  It is not fundamental.
\end{example}

\begin{example}
  For $\g = \mathfrak{sp}_{2n}$, we
  have$ \g = S^2 V$ where $V$ is
  the defining representation for
  $\mathfrak{sp}_{2n}$. Then
  $\theta = 2 \omega_1$, which is
  also not fundamental.
\end{example}

\begin{prop}
  For a simple Lie algebra with
  $\g \ne \mathfrak{sl}_n, \mathfrak{sp}_{2n}$,
  the maximal weight $\theta$ is
  fundamental.
\end{prop}

\begin{example}
  For $\mathfrak{so}_N$ with $N \ge 7$
  (type $B$ or $D$), we have
  $\g = \wedge^2 V = L_{\omega_2}$.
\end{example}

\section{Principal \texorpdfstring{$\mathfrak{sl}_2$}{sl2}-Subalgebra}

\begin{definition}
  Let $\g$ be a simple Lie algebra
  and $\{e_i, f_i, h_i\}$ (where
  $h_i = \alpha_i^\vee)$ be Chevalley
  generators. Let $e = \sum_{i = 1}^r e_i$,
  and $h$ such that $\alpha_i(h) = 2$
  for all $i$ (so $h = 2 \rho^\vee$).
  Note that we have $[h, e] = 2e$ and
  $h = \sum_{i = 1}^r (2 \rho^\vee, \omega_i) \alpha_i^\vee$.
  Let $f = \sum_{i = 1}^r (2 \rho^\vee, \omega_i) f_i$.
  Then $\{h, e, f\}$ spans the
  \emph{principal $\mathfrak{sl}_2$-subalgebra}
  of $\g$.
\end{definition}

\begin{example}
  Let $\g = \mathfrak{sl}_{n + 1}$.
  Then the restriction of the defining
  representation to the principal
  $\mathfrak{sl}_2$ is $L_n$, the
  irreducible representation
  of $\mathfrak{sl}_2$ of highest weight $n$.
\end{example}

\begin{remark}
  Let $\g = \mathfrak{n}_- \oplus \mathfrak{h} \oplus \mathfrak{n}_+$, so
  that $\g = \sum \g[2m]$ where $m$ is the
  height of the corresponding root subspace (and
  $2m$ is the weight with respect to
  $h$).
  Note $\g[0] = \mathfrak{h}$
  and $\dim \g[0] = r$.
  Let $r_m = \dim \g[2m]$.
\end{remark}

\begin{definition}
  We say that $m$ is an
  \emph{exponent} of $\g$ if
  $r_m > r_{m + 1}$. The
  \emph{multiplicity} of an exponent
  $m$ is $r_m - r_{m + 1}$.
\end{definition}

\begin{remark}
  We have $r_0 = r$ and there are
  $r$ exponents (counted with
  multiplicities)
  $m_1 \le m_2 \le \cdots \le m_r$.
  The roots of height $2$
  are given by $\alpha_i + \alpha_j$
  (where $i, j$ are connected in the
  in the Dynkin diagram).
  So $r_0 = r_1 = 1$ and
  $r_2 = r - 1$. Thus $m_1 = 1$
  and $m_2 > 1$. We have
  \[
    m_r = (\rho^\vee, \theta)
    = h_\mathfrak{g} - 1,
  \]
  where $\theta$ is the highest
  root. We call $h_\mathfrak{g}$
  the \emph{Coxeter number} of
  $\g$. Note that
  $\sum_{i = 1}^r m_i = |R_+|$.
\end{remark}

\begin{prop}
  The restriction of $\g$ to its
  principal $\mathfrak{sl}_2$-subalgebra
  decomposes as $\bigoplus_{i = 1}^r L_{2m_i + 1}$.
\end{prop}

\begin{example}
  The exponents for
  $\mathfrak{sl}_n$ are
  $1, 2, \dots, n - 1$.
\end{example}

\begin{definition}
  The \emph{Coxeter number}
  of $\mathfrak{g}$
  is $h_\mathfrak{g} = \langle \theta, \rho^\vee \rangle + 1 = m_r + 1$,
  and the \emph{dual Coxeter number}
  is
  \[
    h_\mathfrak{g}^\vee
    = \langle \widetilde{\theta}^\vee, \rho \rangle + 1,
  \]
  where $\widetilde{\theta}^\vee = 2\theta / (\theta, \theta)$.
  If we normalize $(\theta, \theta) = 2$,
  then $h_\mathfrak{g}^\vee = \frac{1}{2}(\theta, \theta + 2\rho)$, which
  is the eigenvalue of $\frac{1}{2} C$
  (where $C$ is the Casimir operator).
\end{definition}

\section{Complex, Real, and Quaternionic Types}

\begin{definition}
  Let $G$ be a Lie group. An
  irreducible representation $V$ of
  $G$ or $\g$ is of
  \emph{complex type} if
  $V \ncong V^*$, \emph{real type}
  if there exists a symmetric
  isomorphism $V \to V^*$ (i.e. a
  symmetric inner product for $V$), and
  \emph{quaternionic (or symplectic) type}
    if the isomorphism is
    given through an anti-symmetric
    inner product.
\end{definition}

\begin{exercise}
  Let $V$ be an irreducible
  representation of a finite group $G$.
  Show that $\End_{\R G}(V)$
  (i.e. $V \otimes V^*$)
  can only be one of three types:
  \begin{itemize}
    \item complex type if
      $\End_{\R G}(V) \cong \C$,
    \item real type if
      $\End_{\R G}(V) \cong \Mat_{2 \times 2}(\R)$,
    \item quaternionic type if
      $\End_{\R G}(V) \cong \mathbb{H}$.
  \end{itemize}
\end{exercise}

\begin{example}
  Let $L_n$ be an irreducible
  representation of
  $\mathfrak{sl}_2$. Then $L_n$
  is of real type for even $n$ and
  of quaternionic type for odd $n$.
  Thus $L_n = S^n V$ where
  $V = L_1$ is 2-dimensional.
  The invariant form on $S^n V$
  is $S^n B$, where
  $B$ is a skew-symmetric invariant
  form on $V$.
\end{example}

\begin{prop}
  Assume $\lambda = -w_0(\lambda)$,
  so that the corresponding
  representation is of real or
  quaternionic type. Then $L_\lambda$
  is of real type if
  $(2\rho^\vee, \lambda)$ is even
  and of quaternionic type if
  it is odd.
\end{prop}

\begin{proof}
  The number
  $n = (2\rho^\vee, \lambda)$
  is the eigenvalue of $h$ (from the
  principal $\mathfrak{sl}_2$-subalgebra)
  on the highest weight vector.
  Thus we have a decomposition
  \[
    L_\lambda|_{\mathfrak{sl}_2}
    = L_n \oplus \bigoplus_{m < n} k_m L_m,
  \]
  where $L_n$ has multiplicity $1$.
  One can determine the type based
  on $L_n$.
\end{proof}

\section{Review of Compact Lie Groups}

\begin{remark}
  Let $G$ be a real Lie group of
  dimension $n$. Then $\xi \in \wedge^n \g^*$
  gives a generating $n$-form $\omega$,
  which is non-vanishing if $\xi$
  is non-vanishing. This gives
  rise to left- and right-invariant
  measures $\mu_L$ and $\mu_R$
  on $G$, which are
  unique up to a constant.
  We say that $G$ is \emph{unimodular}
  if $\mu_L = \mu_R$ (up to constants).

  When does $\mu_L = \mu_R$?
  For a $1$-dimensional
  representation $V$ of $G$, let $|V|$
  be the representation of $G$ on
  the same space where
  $\rho_{|V|}(g) = |\rho_V(g)|$
  (where $\rho_V : G \to \Aut(V) = \R^\times$).
\end{remark}

\begin{prop}\label{prop:unimodular}
  We have $\mu_L = \mu_R$ if and only
  if $|{\wedge^n \g^*}|$ is a trivial
  representation of $G$.
\end{prop}

\begin{proof}
  We have $\mu_L = \mu_R$ if and only
  if the left-invariant form is right-
  or left-invariant
  up to a sign. This is equivalent
  to $\xi \in \wedge^n \g^*$
  being invariant up to a sign under the
  action of $\g$.
\end{proof}

\begin{prop}
  A compact group is unimodular.
\end{prop}

\begin{proof}
  For compact groups,
  the representation
  $|{\wedge^n \g^*}|$
  gives a continuous homomorphism
  $G \to \R^+$, whose
  only compact subgroup is
  $\{1\}$. The result follows by
  Proposition \ref{prop:unimodular}.
\end{proof}

\begin{prop}
  Let $V$ be an irreducible
  representation of $G$. Then
  $V$ admits a $G$-invariant
  unitary structure.
\end{prop}

\begin{proof}
  Take any positive Hermitian form
  $B$ on $V$, and define
  \[
    B_{\mathrm{av}}(v, w)
    = \int_G B(\rho_V(g) v, \rho_V(g) w)\, dg.
  \]
  This is well-defined and
  invariant by construction.
\end{proof}

\begin{corollary}[Weyl unitary trick]
  Any finite-dimensional
  representation is completely
  reducible.
\end{corollary}

\begin{proof}
  Write $V = W \oplus W^\perp$.
  If $W$ is invariant, then so
  is $W^\perp$.
\end{proof}
