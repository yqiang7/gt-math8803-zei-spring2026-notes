\chapter{Jan.~21 --- Miniscule Weights}

\section{Miniscule Weights}

\begin{remark}
  Let $\g$ be a simple complex
  Lie algebra.
\end{remark}

\begin{definition}
  A dominant integral weight
  $\omega$ for $\g$ is called
  \emph{miniscule} if
  $\langle \omega, \beta \rangle \le 1$
  for every positive coroot $\beta$
  (equivalently, if
  $|\langle \omega, \alpha \rangle| \le 1$
  for any coroot $\beta$).
\end{definition}

\begin{example}
  Clearly $\omega = 0$
  is miniscule.
\end{example}

\begin{example}
  Let $\g = \mathfrak{sl}_n$ with
  fundamental weights
  $\{\omega_i\}_{i = 1}^{n - 1}$,\footnote{Recall a \emph{fundamental weight} is a weight $\omega_i$ such that $\langle \omega_i, \alpha_j^\vee \rangle = \delta_{i, j}$ for all simple coroots $\alpha_j^\vee$.}
  where
  \[
    \omega_i
    = (\underbrace{1, \dots, 1}_{i \text{ ones}}, 0, \dots, 0)
  \]
  Let $\alpha_{i, j} = \alpha_{i, j}^\vee = e_i - e_j$.
  Note that $\langle \omega_i, e_j - e_k \rangle = 0$
  when $j, k \le i$ or $j, k > i$, and
  $\langle \omega_i, e_j - e_k \rangle = 1$
  when $j \le i < k$.
  So all of the $\omega_i$
  are miniscule in this case.
\end{example}

\begin{lemma}
  Every nonzero miniscule weight
  is fundamental.
\end{lemma}

\begin{proof}
  Suppose $\omega$ is miniscule.
  Then there exists $i$ with
  $\langle \omega, \alpha_i^\vee \rangle = 1$.
  Moreover, there can only be one
  such $i$, since if there were
  many, then
  $\langle \omega, \theta^\vee \rangle \ge 2$,
  where $\theta^\vee$ is the longest
  coroot (i.e.
  if $\theta = \sum_{m_i > 0} m_i \alpha_i$ is
  the longest root, then
  $\theta^\vee = \sum_{m_i > 0} m_i \alpha_i^\vee$).
  So $\omega$ is necessarily
  fundamental.
\end{proof}

\begin{example}
  For $G_2$, $F_4$, and $F_8$, none
  of the fundamental weights are
  miniscule.
\end{example}

\begin{lemma}
  A fundamental weight $\omega_i$
  is miniscule if and only if
  $m_i = 1$ where
  $\theta^\vee = \sum_j m_j \alpha_j^\vee$.
\end{lemma}

\begin{proof}
  By the miniscule condition, we
  know $m_i \le 1$. If $m_i = 1$, then for
  any positive coroot $\beta = \sum n_j \alpha_j^\vee$
  we have $n_j \le m_j$, so
  $n_i \le 1$. Thus
  $\langle \omega_i, \beta \rangle = n_i \le 1$, so
  $\omega_i$ is miniscule.
\end{proof}

\begin{lemma}\label{lem:miniscule_zero}
  If $\omega \in Q$ with
  $|\langle \omega, \beta \rangle| \le 1$
  for all coroots $\beta$, then
  $\omega = 0$.
\end{lemma}

\begin{proof}
  Assume to the contrary that
  $\omega = \sum_i \alpha_i \ne 0$.
  We may assume that
  $\sum_i |m_i|$ is smallest possible.
  Then
  $0 < (\omega, \omega) = \sum_{i} m_i(\omega, \alpha_i)$,
  since the form is positive definite.
  Thus there exists $j$ such that
  $m_j$ and $\langle \omega, \alpha_j^\vee \rangle$
  have the same sign.
  By replacing $\omega$ with $-\omega$
  if necessary, we may assume both
  are positive. Then
  $\langle \omega, \alpha_j^\vee \rangle = 1$.
  Consider the reflection
  $s_j(\omega) = \omega - \alpha_j = \sum_i m_i' \alpha_i$.
  So $m_i' = m_j - 1$ and
  $m_i' = m_i$. But then
  $\sum_i |m_i'| = \sum_i |m_i| - 1 < \sum_i |m_i|$,
  contradicting the minimality
  of $\omega$.
\end{proof}

\begin{prop}
  The following conditions are
  equivalent:
  \begin{enumerate}
    \item $\omega$ is miniscule;
    \item all weights of
      $L_\omega$ belong to the
      Weyl orbit $W \omega$;
    \item if $\lambda$ is a dominant
      integral weight such that
      $\omega - \lambda \in Q_+$, then
      $\lambda = \omega$.
  \end{enumerate}
\end{prop}

\begin{proof}
  $(1 \Rightarrow 3)$
  If $\omega = 0$, then
  $-\lambda \in Q_+$, so
  $(\lambda, \rho) \le 0$
  where $\rho = \sum_{i = 1}^r \omega_i$,
  so $\lambda = 0$. Now let
  $\omega = \omega_i$ be miniscule.
  Then $\omega_i - \lambda = \sum_k m_k \alpha_k$
  with $m_k \ge 0$. If $m_k = 0$ for
  $k \ne i$,
  then the problem reduces to a lower
  rank Dynkin diagram. So we can
  assume $m_k > 0$ for every
  $k \ne i$. Let $\beta$ be a positive
  coroot, then
  \[
    \langle \omega_i - \lambda, \beta\rangle
    = \langle \omega_i, \beta \rangle
    - \langle \lambda, \beta \rangle
    \le \langle \omega_i, \beta \rangle
    \le 1.
  \]
  If $\alpha_i^\vee$ does not
  occur in $\beta$, then the above is
  $\le 0$. In particular,
  we have $\langle \omega_i - \lambda, \alpha_j^\vee \rangle \le 0$
  for $j \ne i$. If we also have
  $\langle \omega_i - \lambda, \alpha_i^\vee \rangle \le 0$,
  then $(\omega_i - \lambda, \omega_i - \lambda) \le 0$,
  so $\omega_i = \lambda$. Otherwise,
  $\langle \omega_i - \lambda, \alpha_i^\vee \rangle = 1$.
  Then $m_j > 0$ for every $j$,
  so $\langle \omega_i - \lambda, \theta^\vee \rangle \ge 1$, since
  $\theta^\vee$ is a dominant coweight.
  Then $\langle \lambda, \theta^\vee \rangle \le 0$,
  so we must have
  $\lambda = 0$ since
  $\theta^\vee$ contains all
  $\alpha_j^\vee$ with positive
  coefficients.
  But then $\omega_i \in Q$, which is
  impossible by
  Lemma \ref{lem:miniscule_zero}.

  $(3 \Rightarrow 2)$ If $\mu$ is
  any weight of $L_\omega$, then
  there exists $w \in W$ such that
  $\lambda = w \mu$ is dominant
  (since every orbit of $W$ intersects
  the dominant chamber at exactly
  $1$ point). Then $\omega - \lambda \in Q_+$,
  so $\lambda = \omega$, hence
  $\mu = w^{-1} \omega \in W \omega$.

  $(2 \Rightarrow 1)$ Suppose otherwise
  $\omega$ is not miniscule.
  Then $\langle \omega, \alpha^\vee \rangle > 1$
  for some positive coroot
  $\alpha^\vee$. Then
  \[
    2 (\omega, \alpha)
    > (\alpha, \alpha).
  \]
  Note that $\omega - \alpha$
  is a weight of $L_\omega$ (weight
  of $f_\alpha v_\omega$, where
  $v_\omega$ is a highest weight vector
  and $\{e_\alpha, f_\alpha, \alpha^\vee\}$
  is an $\mathfrak{sl}_2$-triple).
  But $\omega - \alpha$ is not
  $W$-conjugate to $\omega$, since
  \[
    (\omega - \alpha, \omega - \alpha)
    = (\omega, \omega)
    - 2(\omega, \alpha) + (\alpha, \alpha)
    < (\omega, \omega)
  \]
  but the pairing is $W$-invariant.
  Contradiction.
\end{proof}

\begin{corollary}
  If $\omega$ is miniscule, then
  $\chi_\omega = \sum_{\gamma \in W \omega} e^\gamma$.
\end{corollary}

\begin{prop}
  $\omega \in P_+$
  is miniscule if and only if
  the restriction of $L_\omega$
  to any root $\mathfrak{sl}_2$-subalgebra
  of $\g$ is the
  direct sum of $1$-dimensional
  and $2$-dimensional
  representations.
\end{prop}

\begin{proof}
  $(\Rightarrow)$
  Let $\omega$ be miniscule and
  $v \in L_\omega$ the highest
  weight vector (of weight $w \omega$)
  for $(\mathfrak{sl}_2)_\alpha$.
  Then
  \[
    h_\alpha v
    = \langle w \omega, \alpha^\vee \rangle v
    = \langle \omega, w^{-1} \alpha^\vee \rangle v.
  \]
  Then
  $h_\alpha v = 0$ or $h_\alpha v = v$,
  so the representation is
  $1$-dimensional or $2$-dimensional.

  $(\Leftarrow)$ Suppose
  $\omega$ is not miniscule.
  Then there exists $\alpha \in Q_+$ 
  with $\langle \omega, \alpha^\vee \rangle = m > 1$.
  Let $v_\omega$ be a highest
  weight vector, then
  $h_\alpha v_\omega = \langle \omega, \alpha^\vee \rangle v_\omega$, which
  leads to a higher-dimensional
  $\mathfrak{sl}_2$-representation.
\end{proof}

\begin{corollary}
  If $\omega$ is miniscule, then
  for every dominant integral
  weight $\lambda$ of $\g$, we have
  \[
    L_\omega \otimes L_\lambda
    = \bigoplus_{\gamma \in W \omega}
    L_{\lambda + \gamma}.
  \]
  (It is assumed that if
  $\lambda + \gamma$ is not dominant,
  then $L_{\lambda + \gamma} = 0$.)
\end{corollary}

\begin{proof}
  We know
  $\chi_\omega = \sum_{\mu \in W \omega} e^\mu$.
  Then we have
  \[
    \chi_{L_\omega \otimes L_\lambda}
    = \frac{\sum_{\mu \in W \omega} \sum_{w \in W} (-1)^{\ell(\omega)} e^{w(\lambda + \rho) + \mu}}{\Delta}
    = \frac{\sum_{\gamma \in W \omega} \sum_{w \in W} (-1)^{\ell(\omega)} e^{w(\lambda + \gamma + \rho)}}{\Delta}
  \]
  where $\Delta$ is the Weyl denominator.
  If $\lambda + \gamma \notin P_+$,
  then for some $\alpha_i^\vee$,
  we get
  $\langle \lambda + \gamma, \alpha_i^\vee \rangle < 0$. But we know
  $\langle \gamma, \alpha_i^\vee \rangle \ge -1$,
  so $\langle \lambda + \gamma, \alpha_i^\vee \rangle = -1$.
  Thus $\langle \lambda + \gamma + \rho, \alpha_i^\vee \rangle = 0$,
  so for any $w \gamma$,
  the term $w s_i \gamma$ comes
  with the opposite sign.
  So we get that
  \[
    \chi_{L_\omega \otimes L_\lambda}
    = \frac{\sum_{\gamma \in W \omega\, :\, \lambda + \gamma \in P_+}
    \sum_{w \in W} (-1)^{\ell(w)} e^{w(\lambda + \gamma + \rho)}}{\Delta}
    = \sum_{\gamma \in W \omega\, :\, \lambda + \gamma \in P_+}
    \chi_{\lambda + \gamma},
  \]
  which proves the desired result.
\end{proof}

\begin{example}
  For $\mathfrak{sl}_2$, we have
  $L_1 \otimes L_m = L_{m + 1} \oplus L_{m - 1}$,
  which leads to the formula
  \[
    L_m \otimes L_n
    = \bigoplus_{k = |m - n|}^{m + n}
    L_k
  \]
\end{example}

\begin{example}
  Let $V = V_{\omega_1}$ be the defining
  representation for $\GL_n$. Then
  \[
    L_{\omega_1} \otimes L_\lambda
    = \bigoplus_{\mu \in \lambda + \square} L_\mu,
  \]
  where $\lambda$ is a partition
  and $\lambda + \square$ denotes the
  set of partitions obtained by
  adding a single box to $\lambda$.
  For example, for
  $\lambda = (3, 3, 2, 1)$ we have
  \[
    L_{\omega_1} \otimes
    S^{(3, 3, 2, 1)} V
    = S^{(4, 3, 2, 1)} V
    \oplus S^{(3, 3, 3, 1)} V
    \oplus S^{(3, 3, 2, 2)} V
    \oplus S^{(3, 3, 2, 1, 1)} V.
  \]
  Similarly, for
  $\wedge^m V = L_{\omega_m}$
  (where $\omega_m = (1, \dots, 1, 0, \dots, 0)$
  with $m$ ones), we have
  \[
    L_{\omega_m} \otimes L_\lambda
    = \bigoplus_{\mu \in \lambda + m \square}
    L_\mu,
  \]
  where we are allowed to
  add $m$ boxes to $\lambda$
  in $\lambda + m \square$. For
  example,
  \[
    \wedge^2 V
    \otimes S^{(3, 1)} V
    = S^{(4, 2)} V
    \oplus S^{(4, 1, 1)} V
    \oplus S^{(3, 2, 1)} V
    \oplus S^{(3, 1, 1, 1)} V.
  \]
\end{example}
