\chapter{Jan.~12 --- Introduction and Review}

\section{Review and Overview}

\begin{remark}
  Recall that we are interested
  in representations of Lie groups $G$,
  which is closely related to
  representations of Lie algebras
  $\g$.

  We are primarily interested
  in semisimple Lie algebras.
  In this case, we fix a
  \emph{Cartan subalgebra}
  $\mathfrak{h} \subseteq \g$,
  where $r = \dim \mathfrak{h}$
  is called the \emph{rank}.
  We have the Serre generators
  $\{h_i, e_i, f_i\}_{i = 1}^r$
  and relations
  \[
    [h_i, e_j]
    = a_{i, j} e_j, \quad
    [h_i, f_j]
    = a_{i, j} f_j,
    \quad \ad_{e_i}^{1 - a_{i, j}} e_j = 0, \quad
    \ad_{f_i}^{1 - a_{i, j}} f_j = 0,
  \]
  where $a_{i, j} = \langle \alpha_i^\vee, \alpha_j \rangle$
  for $\alpha_i^\vee = 2\alpha_i / (\alpha_i, \alpha_i)$.
  Here $\{\alpha_i\} \subseteq \mathfrak{h}^*$
  and we identify
  $\alpha_i^\vee \leftrightarrow h_i \in \mathfrak{h}$. Then
  \[
    \g = \mathfrak{n}_+
    \oplus \mathfrak{h} \oplus
    \mathfrak{n}_-,
  \]
  where $\mathfrak{n}_+$ is
  generated by $\{e_i\}$ and
  $\mathfrak{n}_-$ is generated
  by $\{f_i\}$. We also have
  \[
    \g = \mathfrak{h} \oplus
    \bigoplus_{\alpha \in R} \g_\alpha,
  \]
  where $R = R_+ \sqcup R_-$.
  We have $R_+ \subseteq Q_+$ and
  $R_- \subseteq Q_-$, where
  $Q_+ = \{\sum_{i = 1}^r n_i \alpha_i : n_i \ge 0\}$.
  If the $a_{i, j}$ are degenerate,
  then we can define
  $\widehat{\g} = \g[t, t^{-1}] \oplus \C c \oplus \C d$,
  where $\C c$ is called the
  \emph{central extension} and
  $d = t \frac{d}{dt}$. We can
  think of these as maps $S^1 \to \g$.

  We can also consider the
  universal enveloping algebra
  $U(\g)$, and the related object.
  $U_q(\g)$ We have an
  $R$-matrix $R_{V, W}$ for
  the representations $V \otimes W$
  and $W \otimes V$, and we have the
  relation
  \[
    R_{1, 2} R_{1, 3} R_{2, 3}
    = R_{2, 3} R_{1, 3} R_{1, 2}
  \]
  in $V_1 \otimes V_2 \otimes V_3$.
  A main goal later in the course will
  be to relate the representations
  of $U_q(\g)$ and
  $\widehat{\g}$.

  In this case, we have the diagram:
  \[
    \begin{tikzcd}
      & & \g \ar[dl] \ar[dr] \\
      & U_q(\g) \ar[dl] \dar[dr] & & \widehat{g} \ar[ll] \ar[dl] \ar[dr]\\
      E_{q, \tau}(\g)
      & & U_q(\widehat{g}) \ar[ll]
      & & \widehat{\widehat{\g}} \ar[ll]
    \end{tikzcd}
  \]
  The object $U_q(\widehat{g})$
  is related to quantum integrable
  models of spin chain type (XXX and XXZ), and $E_{q, \tau}(\g)$
  is the \emph{elliptic quantum group}
  (XYZ).
\end{remark}

\section{Representations of Semisimple Lie Algebras}

\begin{remark}
  Recall the \emph{Weyl group}
  $W = \{s_\alpha(\lambda) = \lambda - \langle \lambda, \alpha^\vee \rangle \alpha\}$.
  The \emph{weight lattice} is
  \[
    P =
    \{\lambda \in E : \langle \lambda, \alpha^\vee \rangle \in \Z, \alpha \in R\}
    = \bigoplus_i \Z \omega_i,
  \]
  where $\omega_i$ are the
  fundamental weights
  satisfying
  $\langle \omega_i, \alpha_j^\vee \rangle = \delta_{i, j}$.

  We can consider the
  \emph{highest weight representation}.
  The \emph{Verma module} is
  $M_\lambda = U(\g) \otimes_{U(\mathfrak{h} \oplus \mathfrak{n}_+)} \C_\lambda$,
  where $\C_\lambda$ is the
  $1$-dimensional representation
  of $U(\mathfrak{h} \oplus \mathfrak{n}_+)$ on which
  $\mathfrak{h}$ acts by
  $\lambda(h)$. Then
  \[
    P(M_\lambda)
    = \lambda - \Q_+,
  \]
  and for each $\lambda \in \mathfrak{h^*}$,
  $M_\lambda$ has a unique
  irreducible quotient $L_\lambda$. The
  \emph{dominant integral weights}
  $\lambda$ satisfy
  \[
    \langle \lambda, \alpha_i^\vee \rangle
    \in \Z_+, \quad 1 \le i \le r,
  \]
  where $\lambda = \sum_{i = 1}^r n_i \omega_i$
  with $n_i \in \Z_+$.
\end{remark}

\begin{theorem}
  The finite-dimensional irreps of
  $\g$ are classified up to isomorphism
  by $\lambda \in P_+$. Moreover,
  $P(V)$ is Weyl invariant, and for
  any $\mu \in P(V)$, $w \in W$,
  \[
    \dim L_\lambda[\mu]
    = \dim L_\lambda[w \mu].
  \]
\end{theorem}

\begin{example}
  For $\g = \mathfrak{sl}_2$, the dominant
  integral weights are
  $n \in \Z_{\ge 0}$, $L_n = V_n$, and
  the Weyl group $W$ acts by reflection.
\end{example}

\begin{remark}[Weyl character formula]
  Let $\chi_V(g) = \tr_V(g)$.
  We can represent $g \sim e^h$, where
  $h \in \mathfrak{h}$. Then
  \[
    \chi_V(e^h)
    = \sum_{\mu \in P} (\dim V(\mu))
    e^{\mu(h)}.
  \]
  We can then formally
  define $\chi_V = \sum_{\mu \in P} (\dim V(\mu)) e^{\mu}$. The
  \emph{Weyl character formula} is
  \[
    \chi_{L_\lambda}
    = \frac{\sum_{w \in W} (-1)^\ell(w) e^{w(\lambda + \rho)}}{\Delta},
  \]
  where $\Delta = \prod_{\alpha \in R_+} (e^{\alpha / 2} - e^{-\alpha / 2}) = \prod_{w \in W} (-1)^{\ell(w) w \rho}$
  is the \emph{Weyl denominator}.
  Here $\rho = \frac{1}{2} \sum_{\alpha \in R_+} \alpha = \sum_{i = 1}^r w_i$.
  The \emph{Weyl dimension formula}
  is then
  \[
    \dim L_\lambda
    = \frac{\prod_{\alpha \in R_+} (\alpha, \lambda + \rho)}{\prod_{\alpha \in R_+} (\alpha, \rho)}.
  \]
  Recall the \emph{Casimir operator}
  $\sum_{i = 1}^{\dim \g} x_i x^i \in U(\g)$,
  which acts by the scalar
  $(\lambda, \lambda + 2\rho)$.
\end{remark}

\section{Representations of \texorpdfstring{$\SL_n$}{SLn} and \texorpdfstring{$\GL_n$}{GLn}}

\begin{prop}
  For general simple $\g$,
  let
  $\lambda = \sum_{i = 1}^r m_i \omega_i$
  be a dominant integral weight.
  Let $T_\lambda = \bigotimes_i L_{\omega_i}^{\otimes m_i}$
  and $v = \bigotimes_i v_{\omega_i}^{\otimes m_i}$.
  Let $V$ be the subrepresentation of
  $T_\lambda$ generated by $v$.
  Then $V \cong L_\lambda$.
\end{prop}

\begin{remark}
  For $\mathfrak{sl}_n$, we have
  $\lambda = \sum_{i = 1}^{n - 1} m_i \omega_i$.
  The Cartan subalgebra is
  \[
    \mathfrak{h}
    = \C^n_0
    = \{(x_1, \dots, x_n) \in \C^n : x_1 + \dots + x_n = 0\}.
  \]
  We have $\alpha_i^\vee = e_i - e_{i - 1}$ and
  $\delta_{i, j} = (\omega_i, \alpha_j^\vee) = (\omega_i, e_j - e_{j + 1})$,
  where $\omega_i = (1, \dots, 1, 0, \dots, 0)$
  with $i$ ones. We
  can associate $\lambda$ with the
  partition
  \[
    \lambda
    = (m_1 + \dots + m_{n - 1},
    m_2 + \dots + m_{n - 1}, \dots, m_{n - 1}, 0)
    = (\lambda_1, \lambda_2, \dots, \lambda_{n - 1}, 0),
  \]
  and $\lambda_1 \ge \lambda_2 \dots \ge \lambda_{n - 1}$.
  Note that $L_{\omega_1}$ is
  the defining
  representation, where
  $v_{\omega_1} = (1, 0, \dots, 0)^T = v_1$, where
  $\{v_1, \dots, v_n\}$ is a
  basis of the defining representation.
  Then we have
  that $L_{\omega_m} = \wedge^m V$
  with highest weight $v_1 \wedge \dots \wedge v_m$.
  Here $e_i = E_{i, i + 1}$.
  Then we see that
  $L_\lambda \subseteq \bigotimes_{i = 1}^{n - 1}(\wedge^i V)^{\otimes m_i}$.
\end{remark}

\begin{remark}
  To move to $\GL_n$, we can write
  \[
    \GL_n(\C)
    = (\C^\times \times \SL_n(\C)) / \mu_n,
  \]
  where $\mu_n$ are the roots of unity
  embedded by
  $z \mapsto (z^{-1}, z I)$.
  We have a covering homomorphism
  \begin{align*}
    \C^\times \times \SL_n(\C)
    &\longrightarrow \GL_n(\C) \\
    (z, A) &\longmapsto zA.
  \end{align*}
  We need to determine the
  holomorphic representations of $\C^\times$.
  Its Lie algebra is spanned by
  $h$ such that $e^{2\pi i h} = 1$.
  Within a representation, $h$
  acts by an operator $H$ such that
  $e^{2\pi i H} = 1$. Thus all
  irreducible representations of
  $\C^\times$ are of the form
  $\chi_N(z) = z^N$.
  So for $\C^\times \times \SL_n(\C)$,
  we have $L_{\lambda, N} = \chi_N \otimes L_\lambda$.
\end{remark}

\begin{exercise}
  Show that if $L_{\lambda, N} = \chi_N \otimes L_\lambda$, then
  $N = nr + \sum_{i = 1}^{n - 1} \lambda_i$
  for some integer $r$.
\end{exercise}

\begin{remark}
  Letting $m_n = r \ge 0$ in the above
  exercise, the
  representation $L_{\lambda, n m_n + \sum_{i = 1}^{n - 1} \lambda_i}$
  for $\gl_n$
  corresponds to the partition
  $(m_1 + \dots + m_n, \dots
    m_{n - 1} + m_n, m_n)$.
\end{remark}

\begin{remark}
  For $\SL_n$, the
  representation $\wedge^n V$ is
  trivial, but it is the determinant
  for $\GL_n$.
  For $\GL_n$, we also have
  $\chi^k$ and $(\chi^*)^k = \chi^{-k}$,
  these are called the
  \emph{polynomial representations}.
\end{remark}

\begin{remark}
  Let $\lambda = (\lambda_1, \dots, \lambda_n)$
  with $\lambda_i \ge \dots \ge \lambda_n$
  be a partition with at most $n$ parts.
  Then $|\lambda| = \sum_i \lambda_i$
  is an eigenvalue of
  $1_n = \sum_{i = 1}^n e_{i, i} \in \gl_n$.
  We can realize 
  $\lambda$ as a Young diagram.
  Note that $L_\lambda$ occurs in
  $V^{\otimes N}$, where
  $V$ is the defining representation.
  We can decompose
  \[
    V^{\otimes N}
    = \bigoplus_{\lambda : |\lambda| = N}
    L_\lambda \otimes \pi_\lambda,
  \]
  where $\pi_\lambda = \Hom_{\GL_n(\C)}(L_\lambda, V^{\otimes N})$.
  There is a natural action of $S_N$
  on $V^{\otimes N}$.
\end{remark}

\begin{theorem}[Schur-Weyl duality]
  Let $A$ be the image of
  $U(\gl_n)$ in $\End(V^{\otimes N})$
  and $B$ be the image of
  $\C S_N$ in $\End(V^{\otimes N})$.
  Then
  \begin{enumerate}
    \item the centralizer of $A$ is
      $B$ and vice versa;
    \item if $\lambda$ has at most
      $n$ parts, then the
      representation $\pi_\lambda$
      of $B$ (and hence of $S_N$)
      is irreducible, and such
      representations are pairwise
      non-isomorphic;
    \item if $\dim V \ge N$, then
      the $\pi_\lambda$ exhaust
      all irreducible representations
      of $S_N$.
  \end{enumerate}
\end{theorem}
