\chapter{Jan.~28 --- Other Classical Lie Algebras, Part 2}

\section{More on Orthogonal Lie Algebras}

\begin{prop}
  For $n \ge 3$, we have
  $\pi_1(\SO_n(\C)) = \Z / 2\Z$.
\end{prop}

\begin{proof}
  There is a deformation retract
  from the surface $X_n$ defined by
  $z_1^2 + \dots + z_n^2 = 1$
  in $\C^n$ to the
  sphere $X_n^\R = X_n \cap \R^n$
  defined by
  $x_1^2 + \dots + x_n^2 = 1$ in $\R^n$:
  Let $\vec{z} = \vec{x} + i \vec{y} \in X_n$ for $\vec{x}, \vec{y} \in \R^n$,
  and note that $|\vec{z}|^2 = 1$
  if and only if $|\vec{x}|^2 - |\vec{y}|^2 = 1$
  and $\vec{x} \cdot \vec{y} = 0$.
  We also have
  \[
    (\vec{x} + t i \vec{y})^2
    = |\vec{x}|^2 - t^2 |\vec{y}|^2
    = 1 + (1 - t^2) |\vec{y}|^2
    \ge 1.
  \]
  So we can define a homotopy
  $f_t : X_n \to X_n$ by
  \[
    f_t(\vec{z})
    = \frac{\vec{x} + t i \vec{y}}{\sqrt{|\vec{x}|^2 - t^2 |\vec{y}|^2}},
  \]
  which satisfies
  $|f_t(z)|^2 = 1$,
  $f_1(z) = z$, and $f_0(z) \in X_n^\R$.
  Now observe that $\SO_n$ acts on
  $X_n$ with fibers isomorphic
  to $\SO_{n - 1}$, so we have a
  long exact sequence
  \[
    \begin{tikzcd}
      \pi_2(X_n)
      \ar[r] & \pi_1(\SO_{n - 1}(\C))
      \ar[r] & \pi_1(\SO_n(\C))
      \ar[r] & \pi_1(X_n).
    \end{tikzcd}
  \]
  The first and last
  groups are trivial for $n \ge 4$,
  so we have
  that $\pi_1(\SO_{n - 1}(\C)) \cong \pi_1(\SO_n(\C))$.
  Thus the result follows once one
  checks that
  $\pi_1(\SO_3(\C)) \cong \Z / 2\Z$
  (left as an exercise).
\end{proof}

\begin{remark}
  Now consider $D_n$, which
  corresponds to $\g = \mathfrak{so}_{2n}$.
  Let $Q = \sum_{i = 1}^n x_i x_{i + n}$.
  The elements of the Cartan
  subalgebra are given by
  $\diag(a_1, \dots, a_n, -a_1, \dots, -a_n)$.
  Let $V$ be the $2n$-dimensional
  defining representation,
  and consider $\wedge^i V$ for
  $1 \le i \le n$. We have $\wedge^i V$
  is irreducible for
  $0 \le i \le n - 1$, and $L_{\omega_i} = \wedge^i V$
  for $1 \le n - 2$. Note
  that $L_{(1, \dots, 1, 0)}$
  is irreducible but not fundamental.
  Letting
  \[
    \omega_{n - 1}
    = (1 / 2, \dots, 1 / 2, 1 / 2)
    \quad \text{and} \quad
    (1 / 2, \dots, 1 / 2, -1 / 2),
  \]
  the corresponding
  $S_+ = L_{\omega_{n - 1}}$
  and $S_- = L_{\omega_n}$
  are the spin representations.
  The characters are
  \[
    \chi_{S_{\pm}}
    = ((x_1^{1 / 2} + x_1^{-1 / 2})\cdots (x_n^{1 / 2} + x_n^{-1 / 2}))_{\pm},
  \]
  where the $\pm$ denotes an even or
  odd number of $-$ signs.
\end{remark}

\begin{example}
  We have $\mathrm{Spin}_4 = \SL_2 \times \SL_2$,
  where factors correspond to
  $S_+$ and $S_-$. We have
  $\mathrm{Spin}_5 = \mathrm{Sp}_4$,
  where $S$ is the $4$-dimensional
  defining representation, and
  $\SO_5 = \mathrm{Sp}_4 / \{\pm 1\}$.
  We have $\mathrm{Spin}_6 = \SL_4$,
  where $S_+, S_-$ are the
  $4$-dimensional defining
  representation and its dual, and
  $\SO_6 = \SL_4 / \{\pm 1\}$.
\end{example}

\begin{example}
  Let $V$ be a finite-dimensional
  vector space, and consider
  $S V = \C[x_1, \dots, x_n]$, where
  $x_1, \dots, x_n$ is an orthonormal
  basis.
  Denote $R^2 = \sum_{i = 1}^n x_i^2 = S^2 V$
  and $\Delta = \sum_{i = 1}^n \partial^2 / \partial x_i^2$.
  Then:
  \begin{enumerate}
    \item Find a first-order
      differential operator making
      $\{R^2, \Delta, \cdot\}$
      an $\mathfrak{sl}_2$-triple.
      Make sure that it commutes
      with the $\SO(V)$ action.
    \item Let $H_m \subseteq S^m V$
      be the subspace of harmonic
      polynomials. Then
      \[
        SV = \bigoplus_{m = 0}^\infty H_m \otimes W_m,
      \]
      where $H_m = L_{m \omega_1}$ is
      the irreducible representation
      of $\SO(V)$, and
      $W_m$ is the Verma module for
      $\mathfrak{sl}_2$ of highest
      weight $m$.
  \end{enumerate}
\end{example}

\section{Clifford Algebras}

\begin{definition}
  Let $V$ be a finite-dimensional
  vector space (over $\mathbb{K} = \R$ or $\C$)
  and $(\cdot, \cdot)$ a non-degenerate
  inner product on $V$. Give an
  associative algebra structure
  to $V$ by
  \[
    v^2 = \frac{1}{2}(v, v).
  \]
  Such an algebra is called a
  \emph{Clifford algebra}, and is
  denoted by $\Cl(V)$.
\end{definition}

\begin{corollary}
  $ab + ba = (a + b)^2 - a^2 - b^2 = (a, b)$.
\end{corollary}

\begin{example}
  The operators
  ${}^i\partial / \partial x_i$
  and $dx_i \wedge \cdot$
  define a Clifford algebra.
\end{example}

\begin{example}
  Let $e^i e^j + e^j e^i = \delta_{i, j}$.
  Then $D = \sum_{i = 1}^n e^i \partial_i$
  (the \emph{Dirac operator})
  satisfies $D^2 = \Delta$.
\end{example}

\begin{theorem}
  The algebra $\Cl(V)$ is isomorphic
  to $\Mat_{2^n}(\mathbb{K})$ if
  $\dim V = 2n$ and to
  $\Mat_{2^n}(\mathbb{K}) \oplus \Mat_{2^n}(\mathbb{K})$
  if $\dim V = 2n + 1$.
\end{theorem}

\begin{proof}
  First consider the even case.
  Choose a basis $a_1, \dots, a_n, b_1, \dots, b_n$
  such that
  \[
    (a_i, a_j)
    = (b_i, b_j) = 0, \quad
    (a_i, b_j) = \delta_{i, j}, \quad
    a_i a_j + a_j a_i = 0, \quad
    b_i b_j + b_j b_i = 0, \quad
    b_i a_i + a_i b_i = 1.
  \]
  Consider $\Cl(V)$-module
  $M = \wedge(a_1, \dots, a_n)$ (note
  that $\dim M = 2^n$) with
  action defined by
  \[
    \rho(a_i) w = a_i w \quad
    \text{and} \quad
    \rho(b_i) w
    = \frac{\partial w}{\partial a_i}.
  \]
  We have the relations
  \[
    1 = \left[a_i, \frac{\partial}{\partial a_i}\right]
    = a_i \frac{\partial}{\partial a_i}
    + \frac{\partial}{\partial a_i} a_i
    \quad \text{and} \quad
    a_j \frac{\partial}{\partial a_i}
    = - \frac{\partial}{\partial a_i} a_j
  \]
  for $i \ne j$. Let
  $c_{I, J} = a_{i_1} \cdots a_{i_k} b_{j_1} \cdots b_{j_m}$
  for $I = \{i_1, \dots, i_k\}$
  and $J = \{j_1, \dots, j_m\}$.
  Check as an exercise that the
  $c_{I, J}$ are linearly
  independent, then
  $\rho : \Cl(V) \to \End(M)$
  is an isomorphism.

  If $\dim V = 2n + 1$, then we
  can pick an extra element $z$
  satisfying
  \[
    (z_, a_i) = (z, b_i) = 0
    \quad \text{and} \quad
    (z, z) = 2,
  \]
  with relations
  $z a_i + a_i z = z b_i + b_i z = 0$
  and $z^2 = 1$.
  Then $zw = \pm (-1)^{\deg w} wz$
  for $w \in M_{\pm}$.
\end{proof}

\begin{remark}
  There is an embedding
  $\mathfrak{so}(V) \to \Cl(V)$.
  Define a map
  \begin{align*}
    \xi : \wedge^2 V = \mathfrak{so}(V)
    &\longrightarrow \Cl(V) \\
    a \wedge b
    &\longmapsto \frac{1}{2} (ab - ba)
    = ab - \frac{1}{2}(a, b).
  \end{align*}
  One can check that
  $[\xi(a \wedge b), \xi(c \wedge d)] = \xi([a \wedge b, c \wedge d])$,
  so $\xi$ is a homomorphism
  of Lie algebras. We have
  $\xi^* M$ for even dimensional
  $V$ and $\xi^* M_{\pm}$ for odd
  dimensional $V$, and
  \[
    \rho_{\xi^* M}(a) = \rho_M(\xi(a))
  \]
  gives $\xi^* M$ the structure of an
  $\mathfrak{so}(V)$-representation
  (and similarly for $\xi^* M_{\pm}$.
  Notice that $\chi^* M$ is reducible:
  \[
    \xi^* M
    = (\xi^* M)_0
    \oplus (\xi^* M)_1
  \]
  as representations, where the
  first factor corresponds to even
  degree
  and the second to odd degree.
\end{remark}

\begin{example}
  We have the following:
  \begin{enumerate}
    \item $(\xi^* M)_0 \cong S_+$
      and $(\xi^* M)_1 \cong S_-$
      for even dimensional $V$.
    \item If $\dim V$ is odd, then
      $\chi^* M_{\pm}$ are both
      isomorphic to $S$.
  \end{enumerate}
\end{example}
