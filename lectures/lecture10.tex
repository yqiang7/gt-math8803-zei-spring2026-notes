\chapter{Feb.~16 --- Real Forms, Part 2}

\section{Compact Real Forms}

\begin{prop}
  Let $\tau$ be the Cartan involution
  (defined in Example \ref{ex:cartan-involution})
  and $\g^c = \g_{(\tau)}$.
  Then the Killing form of $\g^c$ is
  negative definite.
\end{prop}

\begin{proof}
  We can write $\g^c = (\mathfrak{h} \cap \g^c) \oplus \bigoplus_{\alpha \in R_+} (\g_\alpha \oplus \g_{-\alpha}) \cap \g^c$.
  The Killing form is negative definite on
  $\mathfrak{h} \cap \g^c$ since the
  inner product on a coroot lattice is
  positive definite. Thus it is
  negative definite on $\g^c$, as
  $\{i \alpha_j^\vee\}$ is a basis for
  $\g^c \cap \mathfrak{h}$. Now we
  need to show that it is negative definite
  on $(\g_{\alpha} \oplus \g_{-\alpha}) \cap \g^c$.
  Note that for $\g = \mathfrak{sl}_2$,
  we have a basis for $\g^c$ given by
  $ih$, $e - f$, $i(e + f)$, so
  $\g^c = \mathfrak{su}(2)$. So the
  statement holds there. For general
  $\g$, we have that $S_i$ preserves
  $\g^c$, since
  \[
    S =
    \begin{pmatrix}
      0 & 1 \\
      -1 & 0
    \end{pmatrix} \in \SU(2).
  \]
  So we have $\Lie(\SU(2)_i) \subseteq \g^c$.
  For any $w \in W$, the lift $\widetilde{w}$
  preserves $\g^c$, so the restriction
  of the Killing form of $\g^c$ to
  $\g^c \cap (\mathfrak{sl}_2)_\alpha$
  is negative definite.
\end{proof}

\begin{remark}
  Consider $\Aut(\g^c)$. Since the Killing
  form is negative definite,
  $\Aut(\g^c)$ is a closed subgroup of
  $\OO(\g^c)$, so it is compact.
  Moreover, this is a Lie group with
  Lie algebra $\g^c$.
\end{remark}

\begin{corollary}
  $G^c_{\mathrm{ad}} = \Aut(\g^c)^\circ$
  is a connected, compact Lie group with
  Lie algebra $\g^c$.
\end{corollary}

\begin{example}
  We have the following:
  \begin{enumerate}
    \item For $\g = \mathfrak{sl}_n$,
      we have $G_{\mathrm{ad}}^c = \mathrm{PSU}(n) = \SU(n) / \mu_n$,
      where $\mu_n$ is the $n$th roots
      of unity.
    \item For $\g = \mathfrak{so}_{n}$,
      we have $G_{\mathrm{ad}}^c = \SO(n)$
      for odd $n$ and
      $G_{\mathrm{ad}}^c = \SO(n) / \{\pm 1\}$ for even $n$.
    \item For $\g = \mathfrak{sp}_{2n}$,
      we have
      $G_{\mathrm{ad}}^c = \U(n, \mathbb{H}) / \{\pm 1\}$.
      The group
      \[
        \U(n, \mathbb{H})
        = \Sp_{2n}(\C) \cap \U(2n)
      \]
      is called the \emph{quaternionic unitary group}.
  \end{enumerate}
\end{example}

\begin{example}
  We have the following:
  \begin{enumerate}
    \item Consider $A_{n - 1}$,
      which corresponds to the
      split form $\mathfrak{sl}(n, \R)$
      and compact form $\mathfrak{su}(n)$.
      For $n > 2$, we have a quasi-split
      real form as follows: Let
      $s(A) = -J A^T J^{-1}$ where
      $J_{i, j} = (-1)^i \delta_{i, n + 1 - j}$.
      Then
      \[
        e_i, f_i, h_i,
        \longmapsto
        e_{n + 1 - i}, f_{n + 1 - i}, h_{n + 1 - i}.
      \]
      Note that $J$ is a Hermitian or
      skew-Hermitian form of signature $(p, p)$
      with $n = 2p$ or of signature
      $(p + 1, p), (p, p + 1)$,
      which are isomorphic
      when $n = 2p + 1$.

      For $n = 2$, we have
      $\mathfrak{su}(1, 1) = \mathfrak{sl}(2, \R)$
      (and
      $\mathrm{PSU}(1, 1) = \PSL(2, \R)$).\footnote{Note that $\mathrm{PSU}(1, 1)$ is the group of automorphisms of the unit disk, and $\PSL(2, \R)$ is the group of automorphisms of the upper half-plane. The isomorphism comes from the Cayley transform from the unit disk to the upper half-plane.}
    \item For type $B_n$, we have
      compact form $\mathfrak{so}(2n + 1)$
      and split form $\mathfrak{so}(n + 1, n)$.
      There are no nontrivial automorphisms,
      so there are no non-split quasi-split
      forms.

      Particular cases of interest are
      $\SO(3) \cong \SU(2)$ and
      $\SO^+(2, 1) = \mathrm{PSU}(1, 1) = \PSL(2, \R)$.
    \item For type $C_n$, we have the
      split form $\mathfrak{sp}(2n, \R)$ and
      compact form $\mathfrak{u}(n, \mathbb{H})$.
      There are no non-split quasi-split
      reals forms, as there are no
      nontrivial automorphisms
      of the Dynkin diagram.

      Note that $B_2 = C_2$, so we have
      $\mathfrak{so}(3, 2) = \mathfrak{sp}_4(\R)$
      and $\mathfrak{so}(5) = \mathfrak{u}(2, \mathbb{H})$.
    \item For type $D_n$, we have split form
      $\mathfrak{so}(n, n)$ and compact form
      $\mathfrak{so}(2n)$. For $n > 4$,
      there is a unique nontrivial involution,
      while for $n = 4$, we have $\Aut(D) = S_3$.
      However, there is still a unique
      non-split quasi-split form as
      there is only one nontrivial involution
      up to conjugation. Recall
      \[
        A = -J A^T J^{-1}, \quad
        J_{i, j} = \delta_{i, 2n + 1 - j}.
      \]
      Then the quasi-split form
      is given by $J \mapsto J' = gJ$,
      where $g$ permutes $e_n, e_{n + 1}$
      (which corresponds to $\alpha_{n - 1}, \alpha_n$).
      The signature defined by $J'$ is
      $(n + 1, n - 1)$, so the
      quasi-split form is
      $\mathfrak{so}(n + 1, n - 1)$.

      Note that $D_2 = A_1 \oplus A_1$,
      so we have the following isomorphisms:
      \begin{align*}
        \mathfrak{so}(4)
        &= \mathfrak{su}(2) \oplus \mathfrak{su}(2), \\
        \mathfrak{so}(2, 2)
        &= \mathfrak{su}(1, 1) \oplus \mathfrak{su}(1, 1), \\
        \mathfrak{so}(3, 1)
        &= \mathfrak{sl}_2(\C),
      \end{align*}
      where in the last isomorphism we view
      $\mathfrak{sl}_2(\C)$ as a real Lie algebra.

      We also have $D_3 = A_3$, which
      gives the following isomorphisms:
      \begin{align*}
        \mathfrak{so}(6) &= \mathfrak{su}(4), \\
        \mathfrak{so}(3, 3) &= \mathfrak{sl}_4(\R), \\
        \mathfrak{so}(4, 2) &= \mathfrak{su}(2, 2).
      \end{align*}
  \end{enumerate}
\end{example}

\section{Classification of Real Forms}

\begin{remark}
  Write $\g = \g^c \otimes_\R \C$
  and $\omega = \sigma_\tau$ the Cartan
  antilinear involution (so that
  $\g^c$ is the fixed points of $\sigma_\tau$).
  Another real structure on $\g$ is given by
  $\sigma = \omega \circ g$ for
  $g \in \Aut(\g)$, as
  \[
    \sigma^2 = \omega \circ g \circ \omega \circ g
    = 1
  \]
  (note that $\omega \circ \omega = \omega(g)$, and $\omega(g) g = 1$).
  Define
  \[
    (X, Y)
    = \tr(\ad_X \ad_{\omega(Y)}),
  \]
  which is the Hermitian extension of
  the Killing form from $\g^c$ to $\g$.
  Note that $\omega(g) = (\g^\dagger)^{-1}$, where $g^\dagger$ is the
  adjoint of $g$. Thus we see that
  $g$ is self-adjoint.
  \pagebreak

  In particular, $g$ is diagonalizable
  with real eigenvalues. So we can write
  \[
    \g = \bigoplus_{\gamma \in \R}
    \g(\gamma),
  \]
  where $\g(\gamma)$ is the eigenspace of
  $\g$ corresponding to eigenvalue
  $\gamma$. Note that
  \[
    [\g(\beta), \g(\gamma)]
    = \g(\beta \gamma).
  \]
  Consider the operator $|g|^t$ for
  $t \in \R$, which acts on $\g(\gamma)$ as
  $|\gamma|^t$. We can rewrite
  \[
    |g|^t
    = \exp(t \log |g|) \in G_{\mathrm{ad}},
  \]
  which is a $1$-parameter subgroup of $G_{\mathrm{ad}}$.
  Then we can define $\theta := g|g|^{-1}$,
  and we have
  \[
    \begin{cases}
      \theta \circ \omega = \omega \circ \theta, \\
      \theta^2 = 1,
    \end{cases}
  \]
  where the first identity follows from
  $(\theta^\dagger)^{-1} = \theta$.
  We have
  \[
    \theta = |g|^{-1 / 2} g \omega(|g|^{1 / 2}).
  \]
  We can assume $g = \theta$ with
  $\theta \circ \omega = \omega = \theta$,
  or equivalently, that $\theta \in \Aut(\g^c)$
  and $\theta^2 = 1$.
  Thus $\theta$ has $\pm 1$ eigenspaces.
  Note that $\theta'$ defines the same
  real form if and only if
  \[
    \theta' = x \theta (\omega(x))^{-1}
  \]
  for some $x \in \Aut(\g)$. Then
  we have $x \theta(\omega(x))^{-1} = \omega(x) \theta x^{-1}$
  (since $\theta'^2 = 1$, so $\theta'^{-1} = \theta'$).
  Let
  \[
    z = (\omega(x))^{-1} x,
  \]
  so that $\omega(z) = z^{-1}$. Then
  $\theta z = z^{-1} \theta$. Now note that
  $z = x^\dagger x$ is positive definite,
  so if $y = xz^{-1 / 2}$,
  \[
    \omega(y)
    = \omega(x) z^{1 / 2}
    = (x^\dagger)^{-1} z^{1 / 2}
    = x z^{-1 / 2} = y.
  \]
  Thus $y \in \Aut(\g^c)$,
  and we also have that
  \[
    \theta' = x \theta \omega(x)^{-1}
    = x \theta z x^{-1}
    = x z^{-1 / 2} \theta z^{1 / 2} x^{-1}
    = y \theta y^{-1}.
  \]
\end{remark}

\begin{theorem}
  The real forms of $\g$ are in
  one-to-one correspondence with the
  conjugacy classes of involutions
  $\theta \in \Aut(\g^c)$, where
  $\theta \mapsto \omega_\theta = \theta \circ \omega = \omega \circ \theta$.
\end{theorem}

\begin{remark}
  For $\theta : \g \to \g$, denote by
  $\g_\theta$ the corresponding real form.
  For example,
  \[
    \g_1 = \g^c = \g_{(\tau)},
  \]
  where the latter is our old notation
  using the split forms.
\end{remark}

\begin{remark}
  We now have a canonical (up to automorphisms of $\g^c$)
  decomposition
  \[
    \g = \mathfrak{k} \oplus \mathfrak{p},
  \]
  where $\theta = 1$ on $\mathfrak{k}$ and $\theta = -1$ on $\mathfrak{p}$.
  Here $\mathfrak{k}$ is a Lie subalgebra
  and $[\mathfrak{p}, \mathfrak{p}] \subseteq \mathfrak{k}$.
  For $\g^c$ itself, we have
  \[
    \g^c = \mathfrak{k}^c \oplus \mathfrak{p}^c.
  \]
  Moreover, we have
  $\g_\theta = \mathfrak{k}^c \oplus \mathfrak{p}_\theta$,
  where $\mathfrak{p}_\theta = i \mathfrak{p}^c$ (a fixed point of $\sigma = \omega \circ \theta$ has to have an extra $-$ sign, which we can achieve by multiplying by $i$).
\end{remark}

\begin{exercise}
  Show that $\mathfrak{k}$ is reductive
  but not necessarily semisimple.
\end{exercise}
