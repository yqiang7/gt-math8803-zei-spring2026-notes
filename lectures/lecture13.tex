\chapter{Feb.~25 --- Maximal Tori}

\section{Classification of Reductive Lie Groups}

\begin{definition}
  A connected complex Lie group $G$ is
  \emph{reductive} if it has the form
  \[
    ((\C^\times)^r \times G_{\mathrm{ss}}) / Z,
  \]
  where $G_{\mathrm{ss}}$ is semisimple and
  $Z$ is a finite central subgroup.
  A complex Lie group $G$ is called
  \emph{reductive} if $G^0$ is reductive
  and $G / G_0$ is finite.
\end{definition}

\begin{example}
  Recall that
  $\GL_n(\C)
    = (\C^\times \times \SL_n(\C)) / \mu_n$,
  hence $\GL_n(\C)$ is reductive.
\end{example}

\begin{remark}
  With this definition, a simply connected
  complex Lie group with reductive
  Lie algebras is not necessarily reductive
  (e.g. $\C$).
\end{remark}

\begin{remark}
  Given $G = ((\C^\times)^r \times G_{\mathrm{ss}}) / Z$,
  we have $Z \subseteq (S^1)^r \times G_{\mathrm{ss}} \subseteq (\C^\times)^r \times G_{\mathrm{ss}}$.
  Then
  \[
    G^c = ((S^1)^r \times G_{\mathrm{ss}}^c) / Z \subseteq G.
  \]
  Recall that restriction from $G$ to $G^c$
  defines an equivalence of categories between their
  representations, so the representations
  of $G$ are completely reducible and
  the irreps are characterized by
  \[
    (n_1, \dots, n_r, \lambda), \quad
    \lambda \in P_+(G_{\mathrm{ss}}),\, n_i \in \Z.
  \]
  Thus we have the characterize the
  trivial character of $Z$.
\end{remark}

\begin{definition}
  A connected Lie group $G$ is called
  \emph{linear} if it can be realized as
  a subgroup of $\GL_n(\C)$ (or $\GL_n(\R)$)
  for some $n$.
\end{definition}

\begin{remark}
  Any complex semisimple Lie group
  can be realized as a linear one, but
  this is not true for real Lie groups
  (e.g. the universal cover of $\SL_2(\R)$).
\end{remark}

\begin{prop}
  Suppose $\g_\theta$ is a real form of a
  semisimple Lie algebra $\g$, and let
  $G$ be the connected complex Lie group
  corresponding to $\g$. Define
  $G_\theta = G^{\omega_\theta}$.
  Then $G_\theta$ and $G^0_\theta$ are
  linear Lie groups, and every connected
  real semisimple Lie group is of the form
  $G_\theta^0$ for some $\g_\theta$.
\end{prop}

\section{Maximal Tori}

\begin{remark}
  Let $\g$ be a complex semisimple Lie
  algebra and $\g^c$ its compact form.
  Let $G$ be the corresponding connected
  Lie group and $G^c$ its compact part.
  Let $\mathfrak{h}^c \subseteq \g^c$
  be a maximal commutative Lie subalgebra.
  Note that $\mathfrak{h}^c \otimes_\R \C = \mathfrak{h} \subseteq \g$,
  where $\mathfrak{h}$ is
  a Cartan subalgebra in $\g$.
\end{remark}

\begin{remark}
  Recall that we have
  $\{(\mathfrak{h}, \Pi)\}$ (here
  $\mathfrak{h}$ is a Cartan subalgebra
  and $\Pi$ the set of simple roots) are all
  conjugate: For any $(\mathfrak{h}, \Pi)$
  and $(\mathfrak{h}', \Pi')$, there
  exists $g \in G$ such that
  \[
    \Ad_g(\mathfrak{h}, \Pi) = (\mathfrak{h}', \Pi').
  \]
  The same happens for $\g^c$.
\end{remark}

\begin{lemma}
  Any two Cartan subalgebras in $\g^c$
  equipped with systems of simple roots
  are conjugate under some $g \in G^c$.
\end{lemma}

\begin{proof}
  Fix $(\mathfrak{h}^c, \Pi)$ and
  $((\mathfrak{h}^c)', \Pi')$. Then there
  exists $g \in G$ such that
  \[
    \Ad_g(\mathfrak{h}^c, \Pi) = ((\mathfrak{h}^c)', \Pi'),
  \]
  and for $\overline{g} = \omega(g)$,
  we have $\Ad_{\overline{g}}(\mathfrak{h}^c, \Pi) = ((\mathfrak{h}^c)', \Pi')$.
  Then $\overline{g}^{-1} g$ commutes
  with $\mathfrak{h}^c$,
  so it preserves $\Pi$. Now
  $\overline{g} h = g$ for some
  $h \in H = \exp(\mathfrak{h}^c_\C)$.
  Write $g = kp$ for some $k \in G^c$ and
  $p \in P$ by the polar decomposition. Then
  $\overline{g} = kp^{-1}$, so
  $kp^{-1} h = kp$, so $h = p^2$.
  So $p = h^{1 / 2}$ commutes with
  $\mathfrak{h}^c$ and preserves
  $\Pi$. Thus we can take $k \in G^c$
  to be the desired element.
\end{proof}

\begin{definition}
  Given a Cartan subalgebra $\mathfrak{h}^c \subseteq \g^c$,
  the corresponding Lie group
  $H^c = \exp(\mathfrak{h}^c) \subseteq G^c$
  is called a \emph{maximal torus}.
\end{definition}

\begin{corollary}
  Any two maximal tori in $G$ or $G^c$
  equipped with systems of simple roots
  are conjugate.
\end{corollary}

\begin{theorem}
  Every element of a connected compact
  Lie group $K$ is contained in a maximal
  torus, and all maximal tori are
  conjugate.
\end{theorem}

\begin{proof}
  We can assume $K$ is semisimple and
  $K = G^c$. Let $K' \subseteq K$ be the
  set of elements contained in some
  maximal torus. Fix a maximal torus
  $T \subseteq K$, and define
  \begin{align*}
    f : K \times T
    &\longrightarrow K \\
    k, t &\longmapsto k t k^{-1}.
  \end{align*}
  Clearly $\im f = K'$. Note that $K'$ is
  compact, so $K'$ is closed and hence
  $K \setminus K'$ is open. Note that a
  generic
  $x \in \g^c$ is a regular element, i.e. its
  centralizer $\mathfrak{z}_x$ has
  $\dim \mathfrak{z}_x = \rank \g$.
  Then every regular element $g \in K$ is
  contained in the maximal torus
  $\exp(\mathfrak{z}_x)$, so
  the elements of $K' \setminus K$ are
  non-regular. Since the non-regular
  elements are defined by a set of
  polynomial equations, $K' \setminus K$
  is also closed. Hence $K' \setminus K$
  must be empty by connectedness.
\end{proof}

\begin{corollary}
  On compact groups, the exponential map
  $\exp : \g^c \to G^c$ is surjective.
\end{corollary}

\section{Semisimple and Unipotent Elements}

\begin{definition}
  Let $G$ be a connected complex reductive
  Lie group.
  An element $g \in G$ is \emph{semisimple}
  (resp. \emph{unipotent})
  if it acts in every finite-dimensional
  representation of $G$ by semsimple
  (resp. unipotent, i.e. all eigenvalues
  are $1$) operators.
\end{definition}

\begin{exercise}
  Recall that for any compact $G$, there is a faithful
  finite-dimensional representation $Y$.
  Show that $g \in G$
  is semisimple (resp. unipotent) if and only if
  it acts on $Y$ by a semisimple
  (resp. unipotent) operator.
\end{exercise}

\begin{exercise}
  Show that if $G$ is semisimple, then
  the exponential map defines a
  homeomorphism between the set of
  nilpotent elements in $\g = \Lie(G)$ and
  the set of unipotent elements in $G$.
\end{exercise}

\begin{exercise}
  Let $Z$ be the center of a connected
  complex reductive Lie group $G$.
  \begin{enumerate}
    \item Show that the homomorphism
      $\pi : G \to G / Z$ defines a bijection
      between the unipotent elements of
      $G$ and $G / Z$.
    \item Show that the set of semisimple
      elements of $G$ is the preimage under
      $\pi$ of the set of semisimple
      elements in $G / Z$.
  \end{enumerate}
\end{exercise}

\begin{prop}[Jordan decomposition]
  Any $g \in G$ has a unique factorization
  \[
    g = g_s g_u,
  \]
  where $g_s \in G$ is semisimple,
  $g_u \in G$ is unipotent, and
  $g_s g_u = g_u g_s$.
\end{prop}

\begin{proof}
  We can reduce to the adjoint group. Then
  we can decompose
  \[
    \Ad_g = s u
  \]
  uniquely on the level of matrices, where
  $s$ is semisimple and $u$ is unipotent.
  It then suffices to show that
  $s, u$ are $\Ad_{g_s}, \Ad_{g_u}$ for
  some group elements. Fill in the details
  as an exercise.
\end{proof}

\section{Cartan Decomposition}

\begin{remark}
  Let $G$ be a connected complex semisimple
  Lie group. Let $G_\theta \subseteq G$
  be a real form, and write
  $G_\theta = K^c P_\theta$. Then
  $\g_\theta = \mathfrak{k}^c \oplus \mathfrak{p}_\theta$
  and $\g^c = \mathfrak{k}^c \oplus \mathfrak{p}^c$
  where $\mathfrak{p}_\theta = i \mathfrak{p}^c$.
\end{remark}

\begin{prop}
  We have the following:
  \begin{enumerate}
    \item Let $\mathfrak{a}$ be a maximal abelian subspace
      in $\mathfrak{p}_\theta$. Then
      the centralizer $\mathfrak{z}$
      of $\mathfrak{a}$ in $\g^c$
      has the form $\mathfrak{m} \oplus \mathfrak{a}$,
      where $\mathfrak{m}$ is a reductive
      Lie algebra in $\mathfrak{k}^c$.
      Moreover, $\mathfrak{t}$ is
      a Cartan subalgebra in
      $\mathfrak{m}$,
      $\mathfrak{t} \oplus i \mathfrak{a}$
      is a Cartan subalgebra in
      $\mathfrak{t} \oplus \mathfrak{a}$
      is a Cartan subalgebra in $\g_\theta$.
    \item If $a \in \mathfrak{a}$ is
      sufficiently generic in
      $\mathfrak{p}_\theta$, then the
      centralizer of $a$ in
      $\mathfrak{p}_\theta$ is
      $\mathfrak{a}$.
    \item For all $p \in \mathfrak{p}_\theta$,
      there exists $k \in K^c$ such that
      $\Ad_k(p) \in \mathfrak{a}$.
    \item All maximal subspaces of
      $\mathfrak{p}_\theta$ are conjugated
      by $K^c$.
  \end{enumerate}
\end{prop}

\begin{theorem}[Cartan decomposition]\label{thm:cartan-decomposition}
  Let $\mathfrak{a} \subseteq \mathfrak{p}_\theta$
  be a maximal abelian subspace. Let
  $A = \exp(\mathfrak{a}) \subseteq P_\theta \subseteq G_\theta$,
  which is isomorphic to
  $\R^n$ where $n = \dim \mathfrak{a}$.
  Then $G_\theta = K^c A K^c$.
\end{theorem}

\begin{remark}
  The decomposition in Theorem
  \ref{thm:cartan-decomposition}
  is not unique.
\end{remark}

\begin{remark}
  For $G_\theta = \GL_n(\C)$, the
  Cartan decomposition is $U_1 D U_2$
  where $U_1, U_2$ are unitary and
  $D$ is diagonal with positive
  entries, a classical theorem of linear
  algebra.
\end{remark}

\begin{theorem}[E. Cartan]
  Let $G_\theta$ be a real form of a
  connected semisimple complex Lie group.
  Then any compact subgroup $L$ of
  $G_\theta$ is conjugated to a subgroup of $K^c$
  by an element of $P_\theta$. Also,
  every compact subgroup of $G_\theta$
  is contained in a maximal one.
  Therefore, all maximal compact subgroups
  are conjugate to each other.
\end{theorem}
