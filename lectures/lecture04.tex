\chapter{Jan.~26 --- Other Classical Lie Algebras}

\section{Applications of Minuscule Weights, Continued}
\begin{prop}
  We have the following:
  \begin{enumerate}
    \item Let $\lambda$ be a partition of
      $N$. Then
      $\C S_{N + 1} \otimes_{S_{N}} \pi_\lambda
        = \bigoplus_{\mu \in \lambda + \square} \pi_\mu$.
    \item Let $\mu$ be a partition
      of $N + 1$. Then
      $\pi_\mu|_{S_N}
        = \bigoplus_{\lambda \in \mu - \square} \pi_\lambda$.
  \end{enumerate}
\end{prop}

\begin{proof}
  (1) Let $V$ be a vector space of
  sufficiently large dimension.
  By Frobenius reciprocity,
  \[
    \Hom_{S_{N + 1}}(\C S_{N + 1} \otimes_{S_N} \pi_\lambda, V^{\otimes (N + 1)})
    \cong \Hom_{S_N}(\pi_\lambda, V^{\otimes N} \otimes V)
    = V \otimes S^\lambda V.
  \]
  Now by Schur-Weyl duality, we have
  \[
    \Hom_{S_{N + 1}}
    \Big(\bigoplus_{\mu \in \lambda + \square} \pi_\mu, V^{\otimes(N + 1)}\Big)
    = \bigoplus_{\mu \in \lambda + \square} S^\mu V.
  \]
  Since $V \otimes S^\lambda V
    = \bigoplus_{\mu \in \lambda + \square} S^\mu V$,
    we conclude that
    $\C S_{N + 1} \otimes_{S_N} \pi_\lambda
      = \bigoplus_{\mu \in \lambda + \square} \pi_\mu$.

  (2) This is left as an exercise. Use
  a different version of Frobenius reciprocity.
\end{proof}

\begin{definition}
  Let $\lambda$ be a partition, and
  $\lambda^\dagger$ be the
  \emph{conjugate partition}
  (the one corresponding to the
  transposed diagram). For example,
  $(3, 3, 2, 1)^\dagger = (4, 3, 2)$.
\end{definition}

\begin{corollary}
  Let $\C_-$ be the sign representation
  of $S_N$. Then
  $\pi_\lambda \otimes \C_- \cong \pi_{\lambda^\dagger}$.
\end{corollary}

\begin{proof}
  This is left as an exercise.
  The proof is by induction on
  $N = |\lambda|$. Let
  $C = \sum_{i < j} (i\ j)$, and note
  that its eigenvalues are the
  same as the Casimir operator
  of $\SL_N$.
\end{proof}

\begin{prop}[Skew Howe duality]
  We have a decomposition
  $\wedge^n(V \otimes W) = \bigoplus_\lambda S^\lambda V \otimes S^{\lambda^\dagger} W$
  (as $\GL(V) \otimes \GL(W)$-modules).
\end{prop}

\begin{prop}
  Every coset in $P / Q$ contains a
  unique minuscule weight. This gives
  a bijection between $P / Q$ and
  minuscule weights, so the
  number of minuscule weights is
  equal to $\det A$, where $A$ is the
  Cartan matrix.
\end{prop}

\begin{proof}
  Let $C = a + Q \in P / Q$ be a coset.
  Let $\omega \in C \cap P_+$ be the
  element which minimizes
  $\langle \omega, \rho^\vee \rangle$.
  If $\lambda$ is the dominant weight
  for $L_\omega$, then
  $\lambda \in C \cap P_+$ implies
  that
  \[
    (\lambda, \rho^\vee) \ge (\omega, \rho^\vee).
  \]
  Thus $(\omega - \lambda, \rho^\vee) \le 0$, so
  $\omega - \lambda \in Q_+$. Thus
  $\lambda = \omega$, so $\omega$ is
  minuscule. Now suppose $\omega_1, \omega_2 \in C$
  are minuscule and $\omega_1 \ne \omega_2$
  with $\omega_1 - \omega_2 \in Q$.
  By Lemma \ref{lem:minuscule_zero},
  we must have
  $\langle \omega_1 - \omega_2, \beta \rangle \ge 2$
  for all coroots $\beta$. But then
  $\langle \omega_1, \beta \rangle = 1$
  (which implies $\beta > 0$)
  and $\langle \omega_2, \beta \rangle = -1$
  (which implies $\beta < 0$), a
  contradiction.
\end{proof}

\begin{remark}
  Let $A$ be the Cartan matrix.
  For every root, we can write
  \[
    \alpha_i = \sum_{j = 1}^r A_{i, j} \omega_j.
  \]
  We have a covering map
  $\R^r / \Lambda_2 \to \R^r / \Lambda_1$,
  where $\Lambda_2 = P$ and
  $\Lambda_1 = Q$. Then
  $\det A$ is precisely
  the degree of this covering, which
  counts the number of cosets.
\end{remark}

\section{Other Classical Lie Algebras}

\begin{example}
  Recall that $\g = \mathfrak{sp}_{2n}$
  corresponds to the Dynkin
  diagram $C_n$ ($\dynkin{C}{}$),
  where the arrow points from longer
  roots to shorter roots. 
  We have $R_+ = e_i \pm e_j, 2 e_j$.
  The simple roots are
  \[
    \alpha_1 = e_1 - e_2, \quad
    \alpha_2 = e_2 - e_3, \quad
    \ldots, \quad
    \alpha_{n - 1} = e_{n - 1} - e_n, \quad
    \alpha_n = 2e_n.
  \]
  We have $\alpha_i^\vee = \alpha_i$
  for $i \ne n$ and
  $\alpha_n^\vee = e_n$, and
  $\omega_i = (1, \dots, 1, 0, \dots, 0)$
  (with $i$ ones) for $1 \le i \le n$.
\end{example}

\begin{example}
  The Dynkin diagram $B_n$
  ($\dynkin{B}{}$) corresponds to
  $\g = \mathfrak{so}_{2n + 1}$.
  Most things are the same
  as above, but we will have
  $\alpha_n = e_n$ and
  $\alpha_n^\vee = 2en$. We have
  the same $\omega_i$ for $i < n$,
  but we get $\omega_n = (1 / 2, \dots, 1 / 2)$.
  We have $R_+ = e_i \pm e_j, e_i$.
\end{example}

\begin{example}
  The Dynkin diagram
  $D_n$ ($\dynkin{D}{}$)
  corresponds to $\g = \mathfrak{so}_{2n}$.
  In this case we have
  $R_+ = e_i \pm e_j$, and simple
  roots given by
  \[
    \alpha_1 = e_1 - e_2, \quad \dots, \quad
    \alpha_{n - 2} = e_{n - 1},
    \quad \alpha_{n - 1} = e_{n - 1} - e_n,
    \quad \alpha_n = e_{n - 1} + e_n.
  \]
  We have $\omega_i = (1, \dots, 1, 0, \dots, 0)$ (with $i$ ones)
  for $i = 1, \dots, n - 2$, but we get
  $\omega_{n - 1} = (1 / 2, \dots, 1 / 2, 1 / 2)$
  and $\omega_n = (1 / 2, \dots, 1 / 2, -1 / 2)$.
\end{example}

\begin{remark}
  We have the following:
  \begin{itemize}
    \item For $G_2, F_4, F_8$, we have
  $\det A = 1$ (here $A$ is the Cartan
  matrix), so the only minuscule
  weight is $0$.

\item For $B_n$, we have
  $\det A = 2$ (the nontrivial minuscule
  weight is $(1 / 2, \dots, 1 / 2)$,
  and the representation has weights
  $(\pm 1 / 2, \dots, \pm 1 / 2)$ with
  all possible combinations of
  $\pm$ and dimension $2^n$).
    \item For $D_n$, we have
      $\det A = 4$. The minuscule
      weights are $\omega_1, \omega_{n - 1}, \omega_n$.
      Here $\omega_1$ is the
      $2n$-dimensional defining
      representation. The other two
      are spin representations of
      dimension $2^{n - 1}$, with
      weights $(\pm 1 / 2, \dots, \pm 1 / 2)$, taking
      even or odd numbers of $-$ signs.
  \end{itemize}
\end{remark}

\section{Representations of Symplectic Lie Algebras}

\begin{remark}
  For $\g = \mathfrak{sp}_{2n}$,
  we have the Dynkin diagram $C_n$
  and
  \[
    \omega_i = (\underbrace{1, \dots, 1}_{i \text{ ones}}, 0, \dots, 0).
  \]
  The elements of the Cartan subalgebra
  are given by
  $\diag(a_1, \dots, a_n, -a_1, \dots, -a_n)$.
  So $L_{\omega_1} = V$ (the
  defining representation) with
  highest weight $e_1$. Note that
  $\wedge^2 V$ is not irreducible:
  \[
    \wedge^2 V
    = \wedge_0^2 V \oplus \C,
  \]
  where $\C$ is the trivial
  representation spanned by
  $B^{-1} = \sum_i e_{i + n} \wedge e_i$ (note that $B^{-1}$ is
  invariant under $\mathfrak{sp}_{2n}$.
  However, one can check that
  $\wedge_0^2 V$ is irreducible.

  Now let us consider $L_{\omega_j}$
  for $j \ge 2$. Let $B = \sum_i e_i^* \wedge e_{i + n}^*$.
  We have an operator
  \[
    i_B : \wedge^{i + 1} V \longrightarrow \wedge^{i - 1} V,
  \]
  and we can denote
  $\wedge_0^i V = \ker(i_B|_{\wedge^i V})$
  (note that $i_B|_{\wedge^i V}$ is
  injective when $i \ge n$).
  The $\wedge^i_0 V$ are irreducible
  for $i \le n$, and one can check
  that these form all of the
  irreducible representations of
  $\mathfrak{sp}_{2n}$ (compute their
  dimensions and compare them to the
  highest weight representations).

  We can also define an operator
  \begin{align*}
    m_B : \wedge^{i - 1} V &\longrightarrow \wedge^{i + 1} V \\
    u &\mapsto B^{-1} \wedge u.
  \end{align*}
  One can check that $m_B$ and
  $i_B$ together with $h$ (acting as
  $i - n$ on $\wedge^i V$) form
  an $\mathfrak{sl}_2$-triple.
  Then
  \[
    \wedge V
    = \bigoplus_{i = 0}^n L_{\omega_i}
    \otimes L_{n - j}
  \]
  (where $\omega_0 = 0$ and
  $L_{n - j}$ is the
  representation of $\mathfrak{sl}_2$
  of weight $n - j$)
  as representations of
  $\mathfrak{sp}_{2n} \oplus \mathfrak{sl}_2$.
\end{remark}

\section{Representations of Orthogonal Lie Algebras}

\begin{remark}
  First consider $B_n$, which
  corresponds to $\g = \mathfrak{so}_{2n + 1}$.
  Let $Q = \sum_{i = 1}^n x_i x_{i + n} + x^2_{2n + 1}$.
  In this case, the
  Cartan subalgebra is given
  by elements of the form
  $\diag(a_1, \dots, a_n, -a_1, \dots, -a_n, 0)$.
  Let $V$ be the $(2n + 1)$-dimensional
  defining representation. Then
  for $1 \le i \le n - 1$, the
  representation $\wedge^i V$
  is irreducible (one can check this
  using the dimension formula) with
  highest weight
  \[
    \omega_i
    = (\underbrace{1, \dots, 1}_{i \text{ ones}}, 0, \dots, 0).
  \]
  On the other hand, $\wedge^n V$ is
  irreducible but not fundamental,
  with highest weight
  $(1, \dots, 1) = 2\omega_n$.

  Now we consider the spin
  representation $S$ (whose elements
  are called \emph{spinors}).
  It has weights
  \[
    (\pm 1 / 2, \pm 1 / 2, \dots, \pm 1 / 2)
  \]
  (all possible combinations of
  $\pm$). The character of $S$ is
  given by
  \[
    \chi_S(x_1, \dots, x_n)
    = (x_1^{1 / 2} + x_1^{-1 / 2})
    \cdots (x_n^{1 / 2} + x_n^{-1 / 2}).
  \]
\end{remark}

\begin{remark}
  We will want to look
  at the Lie group
  $\mathrm{Spin}_{2n + 1}(\C)$,
  the universal cover of
  $\SO_{2n + 1}(\C)$. For
  $n = 1$, we have
  $\g = \mathfrak{so}_3(\C) = \mathfrak{sl}_2(\C)$.
  We will see that $S$ is
  $2$-dimensional, and
  $\pi_1(\SO_3(\C)) = \Z / 2\Z$.
\end{remark}
