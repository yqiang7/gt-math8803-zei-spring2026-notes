\chapter{Feb.~4 --- Compact Groups}

\section{More on Exponents}

\begin{theorem}[Chevalley's restriction theorem]
  There is an isomorphism
  $\C[\g]^{G} \overset{\cong}{\longrightarrow} \C[\mathfrak{h}]^W$.
\end{theorem}

\begin{theorem}[Harish-Chandra theorem]
  There is an isomorphism
  $\C[\mathfrak{h}]^W \overset{\cong}{\longrightarrow} \mathcal{Z}(U(\g))$.
\end{theorem}

\begin{remark}
  Pick an ordering
  $s_{i_1}, \dots, s_{i_r}$
  of the simple roots. Then
  $c = s_{i_1} \cdots s_{i_r}$
  is the \emph{Coxeter element}, and
  $c^h = 1$ where $h$ is the
  Coxeter number. Then the eigenvalues
  of $c$ are
  $\zeta^{m_i + 1}$ where
  $\zeta = e^{2\pi i / h}$ and
  the $m_i$ are the exponents.
  Also note that $|W| = \prod_{i = 1}^r (m_i + 1)$.

  If $e, f, h$ is the principal
  $\mathfrak{sl}_2$-triple, then one
  can consider $e + \g^f$
  where $\g^f = \ker \ad_f$.
\end{remark}

\section{Matrix Coefficients}

\begin{remark}
  For the rest of this lecture, let
  $G$ be a real compact group and
  $V$ a finite-dimensional
  continuous complex representation
  of $G$.
\end{remark}

\begin{definition}
  A \emph{matrix coefficient}
  of $\rho_V : G \to \GL(V)$
  is a function $G \to \C$ of the
  form
  \[
    g \longmapsto
    \langle f, \rho_V(g) v \rangle
  \]
  for some $v \in V$ and
  $f \in V^*$.
\end{definition}

\begin{prop}
  Matrix coefficients are smooth.
\end{prop}

\begin{proof}
  Call $v \in V$ a smooth vector
  if $\langle f, \rho_V(g) v \rangle$
  is smooth for all $f \in V^*$.
  It is obvious that such vectors
  form a subspace of $V$, call it
  $V_{\mathrm{sm}} \subseteq V$. Fix
  $v \in V$ and
  $\phi : G \to \C$ smooth and
  with compact
  support. Let
  \[
    w
    = w(\phi, v)
    = \int_G \phi(g) \rho_V(g) v\, dg.
  \]
  We claim that $w$ is smooth.
  We have
  \[
    f(\rho(h) w)
    = f\left(\rho_V(h) \int_G \phi(g) \rho_V(g) v\, dg\right)
    = \int_G f(\phi(g) \rho_V(hg) v)\, dg
    = \int_G f(\phi(h^{-1} g)\rho_V(g) v)\, dg.
  \]
  Differentiating under the integral
  sign and noting that
  $\phi(h^{-1} g)$ is smooth in $h$,
  we see that the above expression
  is smooth in $h$. Now choose a
  delta-like sequence $\phi_n$
  with compact support around $1$
  so that
  \[
    \int_G \phi_n(g)\, dg = 1.
  \]
  Then $w_n = w(\phi_n, v) \to v$
  and each $w_n$ is smooth,
  so $v$ is smooth.
\end{proof}

\begin{remark}
  Let $V$ be an irreducible
  representation of $G$. Then:
  \begin{enumerate}
    \item $V$ has an invariant
      positive-definite inner product
      which is unique up to scaling;
    \item one can use an orthonormal
      basis $v_1, \dots, v_n$
      to define matrix coefficients:
      \[
        \psi_{V, i, j}(g)
        = v_j^*(\rho_V(g) v_i)
        = (\rho_V(g) v_i, v_j)
      \]
      (note that this definition is
      independent of normalization).
  \end{enumerate}
\end{remark}

\begin{theorem}[Orthonormality of matrix coefficients]
  Let $V, W$ be irreducible
  representations of $G$.
  \begin{enumerate}
    \item If $V, W$ are not
      isomorphic, then
      \[
        \int_G \psi_{V, i, j}(g)
        \overline{\psi}_{W, k, \ell}(g)\, dg = 0.
      \]
    \item For $V = W$, we have
      \[
        \int_G \psi_{V, i, j}(g)
        \overline{\psi}_{V, k, \ell}(g)\, dg = \frac{\delta_{i, k} \delta_{j, \ell}}{\dim V}.
      \]
  \end{enumerate}
\end{theorem}

\begin{proof}
  Let $\{v_i\}$ and
  $\{w_k\}$ be orthonormal bases
  for $V$ and $W$, respectively. We
  have
  \[
    \int_G \psi_{V, i, j}(g)
    \overline{\psi}_{W, k, \ell}(g)\, dg
    = \int_G ((\rho_V(g) \otimes \rho_{\overline{W}}(g)) (v_i \otimes w_k), v_j \otimes w_\ell)\, dg
  \]
  Define the operator
  \[
    P = \int_G (\rho_V \otimes \rho_{\overline{W}})(g)\, dg
    = \int_G \rho_{V \otimes \overline{W}}(g)\, dg.
  \]
  Since $\overline{W} \cong W^*$,
  we have $P : V \otimes W^* \to V \otimes W^*$.
  Thus
  \[
    \im P
    \subseteq (V \otimes W^*)^G,
  \]
  which is $0$ if $V \ncong W$. On
  the other hand, if $V \cong W$,
  then the only invariant is
  \[
    \vec{u} = \sum_k (v_k \otimes \overline{v}_k),
  \]
  so $P$ is the orthogonal projection
  onto $\vec{u}$. Thus
  \[
    P\vec{x}
    = \frac{(\vec{x}, \vec{u})}{(\vec{u}, \vec{u})} \vec{u},
  \]
  so we have
  $(P(v_i \otimes w_k), v_j \otimes w_\ell) = \delta_{i, j} \delta_{k, \ell} / (\dim V)$.
\end{proof}

\section{Peter-Weyl Theorem}

\begin{theorem}[Peter-Weyl theorem]
  The matrix coefficients
  $\psi_{V, i, j}$ form
  an orthogonal basis in
  $L^2(G)$.
\end{theorem}

\begin{remark}
  Let $V$ be a finite-dimensional
  irrep of $G$. There is a natural
  inclusion
  \begin{align*}
    i_V : V^* &\longhookrightarrow
    \Hom_G(V, L^2(G)), \\
    f &\longmapsto
    [v \mapsto (\rho_{V^*}(\cdot)f)(v)].
  \end{align*}
  \pagebreak
  We claim that $i_V$ is also surjective.
  To see this, let
  $\phi \in \Hom_G(V, L^2(G))$, i.e.
  an $L^2$ function left-invariant
  under $G$. Thus we have that
  \[
    \phi(x)
    = \rho_{V^*}(x g^{-1}) \phi(g)
  \]
  (after modifying $\phi$ on a set
  of measure zero). Setting
  $g = 1$, we get
  $\phi(x) = \rho_{V^*}(x) \phi(1)$,
  so we have
  \[
    \xi : 
    \bigoplus_{V \in \Irr(G)}
    V \otimes V^*
    \cong
    \bigoplus_{V \in \Irr(G)}
    V \otimes \Hom_G(V, L^2(G))
    \longhookrightarrow L^2(G),
  \]
  an embedding of $(G \times G)$-modules.
  Call the left-hand side
  $L^2_{\mathrm{alg}}(G)$.
\end{remark}

\begin{theorem}[Peter-Weyl theorem, alternative]
  $L^2_{\mathrm{alg}}(G)$
  is dense in $L^2(G)$, i.e.
  \[
    L^2(G)
    = \widehat{\bigoplus}_{V \in \Irr(G)}
    V \otimes V^*.
  \]
\end{theorem}

\begin{example}
  Let $G = S^1 = U(1)$. The
  irreducible representations
  of $G$ are $\psi_n(\theta) = e^{i n \theta}$.
  The $e^{i n \theta}$ form a basis
  of $L^2(G) = L^2(S^1)$, where the
  norm is given by
  \[
    \|f\|^2
    = \frac{1}{2\pi}
    \int_0^{2\pi} |f(\theta)|^2\, d\theta.
  \]
  This is the usual Fourier series
  on $S^1$. The Peter-Weyl theorem
  extends this to non-abelian groups.
\end{example}

\begin{exercise}
  Let $G$ be a compact group and
  $H$ a closed subgroup of $G$.
  \begin{enumerate}
    \item Show that
      $L^2(G / H) = \widehat{\bigoplus}_{V \in \Irr(G)} N_H(V) V$, where
      $N_H(V) = \dim V^H$ (the
      space of $H$-invariants).
    \item Let $G = \SO(3)$
      and $H = \SO(2)$. Then show that
      $L^2(G / H) = L^2(S^2) = \widehat{\bigoplus}_{m \ge 0} N_H(m) L_{2m}$,
      and that
      $N_H(m) = 1$ for every $m$.
  \end{enumerate}
\end{exercise}

\section{Introduction to Quantum Mechanics}

\begin{remark}
  Let $\mathcal{H}$ be a Hilbert
  space and $H$ a self-adjoint operator
  on $\mathcal{H}$. The spectrum
  of $H$ gives the \emph{energy levels}
  of the system. The elements
  $\psi(x, y, z) \in L^2(\R^3)$
  are called \emph{wave functions},
  and we assume that they are
  normalized so that $\|\psi\|_{L^2} = 1$.
  This is so that
  \[
    |\psi(x, y, z)|^2 \Delta V
  \]
  gives the probability of a quantum
  particle to be in the region
  $\Delta V$.
  
  In general, there is also a
  time dependence in the wave
  function $\psi$, so we have
  $\psi(x, y, z, t)$. The time
  dependence is governed by the
  Schroedinger equation:
  \[
    i \partial_t \psi = H \psi.
  \]
  One can solve this equation via
  separation of variables, and we
  can write
  \[
    \psi(x, y, z, t)
    = \sum_{N} e^{-i E_N t} \psi_N(x, y, z),
  \]
  where the $\psi_N$ are eigenvectors
  satisfying $H \psi_N = E_N \psi_N$.
\end{remark}

\begin{example}
  For the hydrogen atom,
  we have
  \[
    H = -\frac{1}{2} \Delta - \frac{1}{r},
  \]
  where $\Delta = \partial_x^2 + \partial_y^2 + \partial_z^2$
  is the Laplacian and
  $r = \sqrt{x^2 + y^2 + z^2}$.
  The $\Delta / 2$ is called the
  \emph{kinetic part} of $H$, and
  the $1 / r$ is called the
  \emph{potential part} of $H$.
\end{example}
