\documentclass[12pt, letterpaper, oneside]{book}
\usepackage[margin={0.6in, 0.75in}]{geometry}
\usepackage{microtype}
% \usepackage{kpfonts}
\usepackage{amsmath, amssymb, amsthm}
\usepackage{parskip}
\usepackage[many]{tcolorbox}
\usepackage{footnote}
\usepackage{cancel}
\usepackage{titlesec}
\usepackage{pgffor}
\usepackage[shortlabels, inline]{enumitem}
\usepackage{hyperref}
\usepackage{tikz-cd}
\usepackage{etoolbox}
\usepackage{dynkin-diagrams}

\def\row#1/#2!{#1_{\IfStrEq{#2}{}{n}{#2}} & \dynkin{#1}{#2}\\}
\newcommand{\tble}[1]{
   \renewcommand*\do[1]{\row##1!}
   \[
      \begin{array}{ll}\docsvlist{#1}\end{array}
   \]
}

\usepackage[overload]{textcase}

\renewcommand{\chaptername}{Lecture}
\newtheorem{axiom}{Axiom}[chapter]
\newtheorem{theorem}{Theorem}[chapter]
\newtheorem{prop}{Proposition}[chapter]
\newtheorem{corollary}{Corollary}[theorem]
\newtheorem{lemma}{Lemma}[chapter]
\newtheorem{conjecture}{Conjecture}[theorem]
\theoremstyle{definition}
\newtheorem{definition}{Definition}[chapter]
\newtheorem{exercise}{Exercise}[chapter]
\newtheorem{example}{Example}[definition]
\newtheorem*{remark}{Remark}

\tcbset{sharp corners, breakable, enhanced, parbox=false}
\newtcolorbox{mybox}[3][]
{
  colframe = #2!150,
  colback  = #2!5,
  coltitle = #2!0!white,  
  title    = {#3},
  #1,
}

\titleformat{\chapter}[display]
    {\normalfont\huge\bfseries}{\chaptertitlename\ \thechapter}{20pt}{\Huge}
\titlespacing*{\chapter}{0pt}{0pt}{40pt}

\newcommand{\R}{\mathbb{R}}
\newcommand{\N}{\mathbb{N}}
\newcommand{\Z}{\mathbb{Z}}
\newcommand{\C}{\mathbb{C}}
\newcommand{\Q}{\mathbb{Q}}
\newcommand{\F}{\mathbb{F}}
\newcommand{\K}{\mathbb{K}}
\newcommand{\Ocal}{\mathcal{O}}
\newcommand{\ZZ}{\mathcal{Z}}
\newcommand{\HH}{\mathcal{H}}
\newcommand{\g}{\mathfrak{g}}
\newcommand{\gl}{\mathfrak{gl}}
\newcommand{\slg}{\mathfrak{sl}}
\newcommand{\Mod}[1]{\ {\mathrm{mod}\ #1}}
\newcommand{\Pmod}[1]{\ (\mathrm{mod}\ #1)}

\DeclareMathOperator{\lcm}{lcm}
\DeclareMathOperator{\re}{Re}
\DeclareMathOperator{\im}{Im}
\DeclareMathOperator{\id}{Id}
\DeclareMathOperator{\GL}{GL}
\DeclareMathOperator{\SL}{SL}
\DeclareMathOperator{\SU}{SU}
\DeclareMathOperator{\U}{U}
\DeclareMathOperator{\SO}{SO}
\DeclareMathOperator{\OO}{O}
\DeclareMathOperator{\Sp}{Sp}
\DeclareMathOperator{\B}{B}
\DeclareMathOperator{\Char}{char}
\DeclareMathOperator{\Mat}{Mat}
\DeclareMathOperator{\Hom}{Hom}
\DeclareMathOperator{\Bilin}{Bilin}
\DeclareMathOperator{\rank}{rank}
\DeclareMathOperator{\tr}{tr}
\DeclareMathOperator{\End}{End}
\DeclareMathOperator{\rad}{rad}
\DeclareMathOperator{\sign}{sign}
\DeclareMathOperator{\refl}{refl}
\DeclareMathOperator{\Ind}{Ind}
\DeclareMathOperator{\Fun}{Fun}
\DeclareMathOperator{\Irr}{Irr}
\DeclareMathOperator{\diag}{diag}
\DeclareMathOperator{\triv}{triv}
\DeclareMathOperator{\Path}{Path}
\DeclareMathOperator{\Wt}{Wt}
\DeclareMathOperator{\Span}{Span}
\DeclareMathOperator{\SYT}{SYT}
\DeclareMathOperator{\Vect}{Vect}
\DeclareMathOperator{\Diff}{Diff}
\DeclareMathOperator{\Stab}{Stab}
\DeclareMathOperator{\Ad}{Ad}
\DeclareMathOperator{\ad}{ad}
\DeclareMathOperator{\Lie}{Lie}
\DeclareMathOperator{\Aut}{Aut}
\DeclareMathOperator{\Der}{Der}
\DeclareMathOperator{\PSL}{PSL}
\DeclareMathOperator{\Gr}{Gr}
\DeclareMathOperator{\sym}{sym}
\DeclareMathOperator{\height}{ht}

\title{MATH 8803: Representation Theory II}
\author{Frank Qiang\\Instructor: Anton Zeitlin}
\date{Georgia Institute of Technology\\Spring 2026}

\begin{document}
  \maketitle

  \begingroup
  \let\cleardoublepage\clearpage
  \tableofcontents
  \endgroup

  % \foreach \i in {00, 01, 02, 03, 04, ..., 50} {%
  %   \edef\FileName{lectures/lecture\i.tex}%     The % here are necessary to eliminate any
  %   \IfFileExists{\FileName}{%  spurious spaces that may get inserted
  %      \input{\FileName}%       at these points
  %   }
  % }
  \chapter{Jan.~12 --- Introduction and Review}

\section{Review and Overview}

\begin{remark}
  Recall that we are interested
  in representations of Lie groups $G$,
  which is closely related to
  representations of Lie algebras
  $\g$.

  We are primarily interested
  in semisimple Lie algebras.
  In this case, we fix a
  \emph{Cartan subalgebra}
  $\mathfrak{h} \subseteq \g$,
  where $r = \dim \mathfrak{h}$
  is called the \emph{rank}.
  We have the Serre generators
  $\{h_i, e_i, f_i\}_{i = 1}^r$
  and relations
  \[
    [h_i, e_j]
    = a_{i, j} e_j, \quad
    [h_i, f_j]
    = a_{i, j} f_j,
    \quad \ad_{e_i}^{1 - a_{i, j}} e_j = 0, \quad
    \ad_{f_i}^{1 - a_{i, j}} f_j = 0,
  \]
  where $a_{i, j} = \langle \alpha_i^\vee, \alpha_j \rangle$
  for $\alpha_i^\vee = 2\alpha_i / (\alpha_i, \alpha_i)$.
  Here $\{\alpha_i\} \subseteq \mathfrak{h}^*$
  and we identify
  $\alpha_i^\vee \leftrightarrow h_i \in \mathfrak{h}$. Then
  \[
    \g = \mathfrak{n}_+
    \oplus \mathfrak{h} \oplus
    \mathfrak{n}_-,
  \]
  where $\mathfrak{n}_+$ is
  generated by $\{e_i\}$ and
  $\mathfrak{n}_-$ is generated
  by $\{f_i\}$. We also have
  \[
    \g = \mathfrak{h} \oplus
    \bigoplus_{\alpha \in R} \g_\alpha,
  \]
  where $R = R_+ \sqcup R_-$.
  We have $R_+ \subseteq Q_+$ and
  $R_- \subseteq Q_-$, where
  $Q_+ = \{\sum_{i = 1}^r n_i \alpha_i : n_i \ge 0\}$.
  If the $a_{i, j}$ are degenerate,
  then we can define
  $\widehat{\g} = \g[t, t^{-1}] \oplus \C c \oplus \C d$,
  where $\C c$ is called the
  \emph{central extension} and
  $d = t \frac{d}{dt}$. We can
  think of these as maps $S^1 \to \g$.

  We can also consider the
  universal enveloping algebra
  $U(\g)$, and the related object.
  $U_q(\g)$ We have an
  $R$-matrix $R_{V, W}$ for
  the representations $V \otimes W$
  and $W \otimes V$, and we have the
  relation
  \[
    R_{1, 2} R_{1, 3} R_{2, 3}
    = R_{2, 3} R_{1, 3} R_{1, 2}
  \]
  in $V_1 \otimes V_2 \otimes V_3$.
  A main goal later in the course will
  be to relate the representations
  of $U_q(\g)$ and
  $\widehat{\g}$.

  In this case, we have the diagram:
  \[
    \begin{tikzcd}
      & & \g \ar[dl] \ar[dr] \\
      & U_q(\g) \ar[dl] \dar[dr] & & \widehat{g} \ar[ll] \ar[dl] \ar[dr]\\
      E_{q, \tau}(\g)
      & & U_q(\widehat{g}) \ar[ll]
      & & \widehat{\widehat{\g}} \ar[ll]
    \end{tikzcd}
  \]
  The object $U_q(\widehat{g})$
  is related to quantum integrable
  models of spin chain type (XXX and XXZ), and $E_{q, \tau}(\g)$
  is the \emph{elliptic quantum group}
  (XYZ).
\end{remark}

\section{Representations of Semisimple Lie Algebras}

\begin{remark}
  Recall the \emph{Weyl group}
  $W = \{s_\alpha(\lambda) = \lambda - \langle \lambda, \alpha^\vee \rangle \alpha\}$.
  The \emph{weight lattice} is
  \[
    P =
    \{\lambda \in E : \langle \lambda, \alpha^\vee \rangle \in \Z, \alpha \in R\}
    = \bigoplus_i \Z \omega_i,
  \]
  where $\omega_i$ are the
  fundamental weights
  satisfying
  $\langle \omega_i, \alpha_j^\vee \rangle = \delta_{i, j}$.

  We can consider the
  \emph{highest weight representation}.
  The \emph{Verma module} is
  $M_\lambda = U(\g) \otimes_{U(\mathfrak{h} \oplus \mathfrak{n}_+)} \C_\lambda$,
  where $\C_\lambda$ is the
  $1$-dimensional representation
  of $U(\mathfrak{h} \oplus \mathfrak{n}_+)$ on which
  $\mathfrak{h}$ acts by
  $\lambda(h)$. Then
  \[
    P(M_\lambda)
    = \lambda - \Q_+,
  \]
  and for each $\lambda \in \mathfrak{h^*}$,
  $M_\lambda$ has a unique
  irreducible quotient $L_\lambda$. The
  \emph{dominant integral weights}
  $\lambda$ satisfy
  \[
    \langle \lambda, \alpha_i^\vee \rangle
    \in \Z_+, \quad 1 \le i \le r,
  \]
  where $\lambda = \sum_{i = 1}^r n_i \omega_i$
  with $n_i \in \Z_+$.
\end{remark}

\begin{theorem}
  The finite-dimensional irreps of
  $\g$ are classified up to isomorphism
  by $\lambda \in P_+$. Moreover,
  $P(V)$ is Weyl invariant, and for
  any $\mu \in P(V)$, $w \in W$,
  \[
    \dim L_\lambda[\mu]
    = \dim L_\lambda[w \mu].
  \]
\end{theorem}

\begin{example}
  For $\g = \mathfrak{sl}_2$, the dominant
  integral weights are
  $n \in \Z_{\ge 0}$, $L_n = V_n$, and
  the Weyl group $W$ acts by reflection.
\end{example}

\begin{remark}[Weyl character formula]
  Let $\chi_V(g) = \tr_V(g)$.
  We can represent $g \sim e^h$, where
  $h \in \mathfrak{h}$. Then
  \[
    \chi_V(e^h)
    = \sum_{\mu \in P} (\dim V(\mu))
    e^{\mu(h)}.
  \]
  We can then formally
  define $\chi_V = \sum_{\mu \in P} (\dim V(\mu)) e^{\mu}$. The
  \emph{Weyl character formula} is
  \[
    \chi_{L_\lambda}
    = \frac{\sum_{w \in W} (-1)^\ell(w) e^{w(\lambda + \rho)}}{\Delta},
  \]
  where $\Delta = \prod_{\alpha \in R_+} (e^{\alpha / 2} - e^{-\alpha / 2}) = \prod_{w \in W} (-1)^{\ell(w) w \rho}$
  is the \emph{Weyl denominator}.
  Here $\rho = \frac{1}{2} \sum_{\alpha \in R_+} \alpha = \sum_{i = 1}^r w_i$.
  The \emph{Weyl dimension formula}
  is then
  \[
    \dim L_\lambda
    = \frac{\prod_{\alpha \in R_+} (\alpha, \lambda + \rho)}{\prod_{\alpha \in R_+} (\alpha, \rho)}.
  \]
  Recall the \emph{Casimir operator}
  $\sum_{i = 1}^{\dim \g} x_i x^i \in U(\g)$,
  which acts by the scalar
  $(\lambda, \lambda + 2\rho)$.
\end{remark}

\section{Representations of \texorpdfstring{$\SL_n$}{SLn} and \texorpdfstring{$\GL_n$}{GLn}}

\begin{prop}
  For general simple $\g$,
  let
  $\lambda = \sum_{i = 1}^r m_i \omega_i$
  be a dominant integral weight.
  Let $T_\lambda = \bigotimes_i L_{\omega_i}^{\otimes m_i}$
  and $v = \bigotimes_i v_{\omega_i}^{\otimes m_i}$.
  Let $V$ be the subrepresentation of
  $T_\lambda$ generated by $v$.
  Then $V \cong L_\lambda$.
\end{prop}

\begin{remark}
  For $\mathfrak{sl}_n$, we have
  $\lambda = \sum_{i = 1}^{n - 1} m_i \omega_i$.
  The Cartan subalgebra is
  \[
    \mathfrak{h}
    = \C^n_0
    = \{(x_1, \dots, x_n) \in \C^n : x_1 + \dots + x_n = 0\}.
  \]
  We have $\alpha_i^\vee = e_i - e_{i - 1}$ and
  $\delta_{i, j} = (\omega_i, \alpha_j^\vee) = (\omega_i, e_j - e_{j + 1})$,
  where $\omega_i = (1, \dots, 1, 0, \dots, 0)$
  with $i$ ones. We
  can associate $\lambda$ with the
  partition
  \[
    \lambda
    = (m_1 + \dots + m_{n - 1},
    m_2 + \dots + m_{n - 1}, \dots, m_{n - 1}, 0)
    = (\lambda_1, \lambda_2, \dots, \lambda_{n - 1}, 0),
  \]
  and $\lambda_1 \ge \lambda_2 \dots \ge \lambda_{n - 1}$.
  Note that $L_{\omega_1}$ is
  the defining
  representation, where
  $v_{\omega_1} = (1, 0, \dots, 0)^T = v_1$, where
  $\{v_1, \dots, v_n\}$ is a
  basis of the defining representation.
  Then we have
  that $L_{\omega_m} = \wedge^m V$
  with highest weight $v_1 \wedge \dots \wedge v_m$.
  Here $e_i = E_{i, i + 1}$.
  Then we see that
  $L_\lambda \subseteq \bigotimes_{i = 1}^{n - 1}(\wedge^i V)^{\otimes m_i}$.
\end{remark}

\begin{remark}
  To move to $\GL_n$, we can write
  \[
    \GL_n(\C)
    = (\C^\times \times \SL_n(\C)) / \mu_n,
  \]
  where $\mu_n$ are the roots of unity
  embedded by
  $z \mapsto (z^{-1}, z I)$.
  We have a covering homomorphism
  \begin{align*}
    \C^\times \times \SL_n(\C)
    &\longrightarrow \GL_n(\C) \\
    (z, A) &\longmapsto zA.
  \end{align*}
  We need to determine the
  holomorphic representations of $\C^\times$.
  Its Lie algebra is spanned by
  $h$ such that $e^{2\pi i h} = 1$.
  Within a representation, $h$
  acts by an operator $H$ such that
  $e^{2\pi i H} = 1$. Thus all
  irreducible representations of
  $\C^\times$ are of the form
  $\chi_N(z) = z^N$.
  So for $\C^\times \times \SL_n(\C)$,
  we have $L_{\lambda, N} = \chi_N \otimes L_\lambda$.
\end{remark}

\begin{exercise}
  Show that if $L_{\lambda, N} = \chi_N \otimes L_\lambda$, then
  $N = nr + \sum_{i = 1}^{n - 1} \lambda_i$
  for some integer $r$.
\end{exercise}

\begin{remark}
  Letting $m_n = r \ge 0$ in the above
  exercise, the
  representation $L_{\lambda, n m_n + \sum_{i = 1}^{n - 1} \lambda_i}$
  for $\gl_n$
  corresponds to the partition
  $(m_1 + \dots + m_n, \dots
    m_{n - 1} + m_n, m_n)$.
\end{remark}

\begin{remark}
  For $\SL_n$, the
  representation $\wedge^n V$ is
  trivial, but it is the determinant
  for $\GL_n$.
  For $\GL_n$, we also have
  $\chi^k$ and $(\chi^*)^k = \chi^{-k}$,
  these are called the
  \emph{polynomial representations}.
\end{remark}

\begin{remark}
  Let $\lambda = (\lambda_1, \dots, \lambda_n)$
  with $\lambda_i \ge \dots \ge \lambda_n$
  be a partition with at most $n$ parts.
  Then $|\lambda| = \sum_i \lambda_i$
  is an eigenvalue of
  $1_n = \sum_{i = 1}^n e_{i, i} \in \gl_n$.
  We can realize 
  $\lambda$ as a Young diagram.
  Note that $L_\lambda$ occurs in
  $V^{\otimes N}$, where
  $V$ is the defining representation.
  We can decompose
  \[
    V^{\otimes N}
    = \bigoplus_{\lambda : |\lambda| = N}
    L_\lambda \otimes \pi_\lambda,
  \]
  where $\pi_\lambda = \Hom_{\GL_n(\C)}(L_\lambda, V^{\otimes N})$.
  There is a natural action of $S_N$
  on $V^{\otimes N}$.
\end{remark}

\begin{theorem}[Schur-Weyl duality]
  Let $A$ be the image of
  $U(\gl_n)$ in $\End(V^{\otimes N})$
  and $B$ be the image of
  $\C S_N$ in $\End(V^{\otimes N})$.
  Then
  \begin{enumerate}
    \item the centralizer of $A$ is
      $B$ and vice versa;
    \item if $\lambda$ has at most
      $n$ parts, then the
      representation $\pi_\lambda$
      of $B$ (and hence of $S_N$)
      is irreducible, and such
      representations are pairwise
      non-isomorphic;
    \item if $\dim V \ge N$, then
      the $\pi_\lambda$ exhaust
      all irreducible representations
      of $S_N$.
  \end{enumerate}
\end{theorem}

  \chapter{Jan.~14 --- Applications of Schur-Weyl Duality}

\section{The Schur Functor}

\begin{remark}
  Let $V$ be the defining
  representation for $\GL_n$. Then
  \[
    V^{\otimes N}
    = \bigoplus_{\lambda : |\lambda| = N} L_\lambda \otimes \pi_\lambda.
  \]
  Recall that if
  $\lambda = (\lambda_1, \dots, \lambda_n)$, then
  we have
  \[
    \lambda_1 = m_1 + \dots + m_n,
    \quad \lambda_2 = m_2 + \dots + m_n, \quad \dots, \quad
    \lambda_n = m_n.
  \]
\end{remark}

\begin{definition}
  Suppose we are given the
  partition $\lambda$ of $N$.
  The \emph{Schur functor} $S^\lambda$
  is given by
  \[
    S^\lambda V
    = \Hom_{S_N}(\pi_\lambda, V^{\otimes N})
  \]
  for a vector space $V$.
  Note that this language, we have
  $V^{\otimes N} = \bigoplus_\lambda S^\lambda V \otimes \pi_\lambda$.
\end{definition}

\begin{example}
  Consider the following:
  \begin{enumerate}
    \item $S^{(n)} V = S^n V$, where
      $(n)$ is the partition
      of $n$ with a single part.
    \item $S^{(1^n)} V = \wedge^n V$, where
      $(1^n)$ is the partition of $n$
      with $n$ parts equal to $1$.
    \item $V \otimes V = S^{(2)} V \otimes \C_+ \oplus S^{(1, 1)} V \otimes \C_-$,
      where $\C_2$ acts trivially on
      $\C_+$ and
      by the sign on $\C_-$.
    \item $V \otimes V \otimes V = S^{(3)} V \otimes \C_+ \oplus S^{(2, 1)} V \otimes \C^2 \oplus S^{(1, 1, 1)} V \otimes \C_-$,
      where $S_3$ acts trivially
      on $\C_+$ and by sign on
      $\C_-$ as before, and
      $\C^2 = \{(x, y, z) : x + y + z = 0\}$.

      Note that $V \otimes V = S^2 V \oplus \wedge^2 V$, so
      $S^2 V \otimes V = S^3 V \oplus S^{(2, 1)} V$ and
      $\wedge^2 V \otimes V = \wedge^3 V \oplus S^{(2, 1)} V$.
  \end{enumerate}
\end{example}

\begin{remark}
  Let $\dim V = N$ and $\lambda$ have
  $k$ parts.
  Recall that by the Weyl dimension
  formula,
  \[
    \dim L_\lambda
    = \frac{\prod_{\alpha \in R_+} (\alpha, \lambda + \rho)}{\prod_{\alpha \in R_+} (\alpha, \rho)}.
  \]
  We have
  $R_+ = \{\alpha_{i, j} = e_i - e_j : i < j\}$
  and $\rho = \sum_{i = 1}^{N - 1} \omega_i = (N - 1, N - 2, \dots, 1, 0)$
  (recall that $\omega_i$ is $i$
  ones followed by zeros). Thus we
  see that
  \[
    \dim S^\lambda V
    = \prod_{1 \le i < j \le N} \frac{\lambda_i - \lambda_j + j - i}{j - i}
    = \prod_{1 \le i < j \le k} \frac{\lambda_i - \lambda_j + j - i}{j - i}
    \prod_{1 \le i < k < j \le N}
    \frac{\lambda_i + j - i}{j - i}.
  \]
  We can rewrite the second product as
  \[
    \prod_{1 \le < k < j \le N}
    \frac{\lambda_i + j - i}{j - i}
    = \prod_{i = 1}^k
    \frac{(N + 1 - i) \cdots (N + \lambda_i - i)}{(k + 1 - i) \cdots (k + \lambda_i - i)}.
  \]
\end{remark}

\begin{prop}
  We have $\dim S^\lambda V = P_\lambda(N)$,
  where $P_\lambda$ is a polynomial
  of degree $|\lambda|$ with rational
  coefficients and integer roots.
  The roots of $P_\lambda$ are all
  integers from the interval
  $[1 - \lambda_1, k - 1]$
  (occurring with multiplicities).
\end{prop}

\begin{example}
  Let $P_n(N)$ correspond to $S^n V$.
  Then $\lambda_1 = n$ and $k = 1$, and
  \[
    P_n(N)
    = \dim S^n V
    = \binom{N + n - 1}{n}.
  \]
  Similarly, one can see that
  \[
    P_{1^n}(N)
    = \dim \wedge^n V = \binom{N}{n}.
  \]
  One can also consider
  $P_{(a, b)}(N)$ corresponding
  to partitions with
  two parts.
  The values $P_{(a, n)}(N)$
  are called the
  Narayana numbers, which are of use
  in combinatorics.
\end{example}

\section{Invariant Theory}

\begin{remark}
  Let $V$ be a finite-dimensional
  vector space and
  $\{T_i\} \in (V^*)^{\otimes m_i} \otimes V^{\otimes n_i}$
  for $i = 1, \dots, k$. One would
  like to characterize \emph{invariants}
  of such collections, i.e.
  polynomial functions
  $F(T_1, \dots, T_k)$ which are
  invariant under the action of
  $\GL(V)$.

  One can think of such a tensor
  in $(V^*)^{\otimes m_i} \otimes V^{\otimes n_i}$
  as a vertex with $m_i$ incoming
  edges and $n_i$ outgoing edges.
  Then constructing invariants
  $\{T_i\}$ reduces to studying
  graphs where $T_i$ corresponds to
  a vertex $v_i$ of the graph $\Gamma$.
  This allows us to assign to a given
  graph an invariant function
  $F_{\Gamma}$.
\end{remark}

\begin{theorem}
  The functions $F_\Gamma$
  for various $\Gamma$ span the
  space of invariant functions.
\end{theorem}

\begin{proof}
  We can view an invariant as an
  invariant element of the space
  $\bigotimes_{i = 1}^k ((V^*)^{\otimes m_i} \otimes  V^{\otimes n_i})$,
  which we can view as
  $\End_{\GL(n)}(V^{\otimes M}, V^{\otimes N})$,
  where $M = \sum d_i m_i$ (the number
  of incoming edges) and
  $N = \sum d_i n_i$ (the number of
  outgoing edges).
  Note that this space is empty
  when $M \ne N$, and the
  statement follows by Schur-Weyl
  duality when $M = N$.
\end{proof}

\begin{example}
  Let $m_i = n_i = 1$. Then
  $T_1, \dots, T_k$ are matrices.
  Then the graph $\Gamma$ must look
  like a cycle, hence the invariants
  are all of the form
  \[
    F_{j_1, \dots, j_r}
    (T_1, \dots, T_k)
    = \tr(T_{j_1} \cdots T_{j_r}).
  \]
  Note that these invariants
  are asymptotically algebraically
  independent (when $V$ is large enough).
  In particular,
  if $P(T_1, \dots, T_k) = 0$ in
  all dimensions, then
  $\tr(P(T_1, \dots, T_k) T_{k + 1}) = 0$,
  which cannot be true as the trace
  decomposes in terms of
  the $F_{j_1, \dots, j_r}$.
  (However, note that
  $[X, Y] = 0$ for $1 \times 1$
  matrices and
  $[Z, [X, Y]^2] = 0$ for
  $2 \times 2$ matrices.)
  This also implies the uniqueness
  of the $\mu_n$ in the BCH formula:
  \[
    \log (\exp(x) \exp(y))
    = \sum_{n \ge 1} \frac{\mu_n(x, y)}{n!}.
  \]
\end{example}

\section{Weyl Character Formula for \texorpdfstring{$\GL_n$}{GLn}}

\begin{remark}[Weyl character formula for $\GL_n$]
  Recall that Weyl's character formula
  gives
  \[
    \chi_{\lambda}
    = \frac{\sum_{w \in W} (-1)^{\ell(w)} e^{w(\lambda + \rho)}}
    {\sum_{\alpha \in R_+} (e^{\alpha / 2} - e^{-\alpha / 2})}, \tag{$*$}
  \]
  where the denominator
  is $\Delta = \prod_{\alpha \in R_+} (e^{\alpha / 2} - e^{-\alpha / 2}) = \prod_{w \in W} (-1)^{\ell(w)} e^{w(\rho)}$.
  Letting $\rho = \frac{1}{2} \sum_{\alpha \in R_+} \alpha$,
  \[
    \Delta = e^\rho \prod_{\alpha \in R_+} (1 - e^{-\alpha})
    = x_1^{n - 1} x_2^{n - 2} \cdots x_n^0
    \prod_{i < j} (1 - x_j / x_i),
  \]
  where $\rho = (n - 1, n - 2, \dots, 1, 0)$
  and $x_i = e^{e_i}$ (e.g.
  $x_1 = e^{(1, 0, \dots, 0)}$).
  After multiplying we get that
  \[
    \Delta = \prod_{i < j} (x_i - x_j).
  \]
  On the other hand, using
  $\Delta = \prod_{w \in W} (-1)^{\ell(w)} e^{w(\rho)}$,
  we have
  \[
    \Delta = \sum_{w \in W} (-1)^{\ell(w)} e^{w(\rho)}
    = \sum_{s \in S_n} \sign(s) x^{n - 1}_{s(1)} \cdots x^0_{s(n)}.
  \]
  Comparing these two formulas,
  we recover the formula for the
  Vandermonde determinant:
  \[
    \det(\{x_j^{n - i}\}_{1 \le i, j \le n})
    = \sum_{s \in S_n} \sign(s) x^{n - 1}_{s(1)} \cdots x^0_{s(n)}
    = \prod_{i < j} (x_i - x_j).
  \]
  Now applying this to the numerator of
  $(*)$, we have
  \[
    \sum_{w \in W} (-1)^{\ell(w)} e^{w(\lambda + \rho)}
    = \sum_{s \in S_n} \sign(s) x^{\lambda_1 + n - 1}_{s(1)} \cdots x^{\lambda_n + 0}_{s(n)}.
  \]
  Thus in total, the
  character $\chi_{\lambda}$ is given by
  \[
    \chi_{\lambda}
    = \frac{\sum_{s \in S_n} \sign(s) x^{\lambda_1 + n - 1}_{s(1)} \cdots x^{\lambda_n + 0}_{s(n)}}
    {\prod_{i < j} (x_i - x_j)}
    = \frac{\det(\{x_i^{\lambda_j + n - i}\})}{\prod_{i < j} (x_i - x_j)}.
  \]
  These functions are known as the
  \emph{Schur polynomials}
  $s_\lambda(x_1, \dots, x_n)$.
\end{remark}

\begin{example}[Character of $S^{(n)} V$]
  Using the above formula, we get the identity
  \[
    s_{(m)}(x_1, \dots, x_n)
    = \sum_{1 \le j_1 \le \cdots \le j_m \le n} x_{j_1} \cdots x_{j_m}
    = h_m(x_1, \dots, x_m),
  \]
  the $m$th complete symmetric function.
\end{example}

\begin{example}[Character of $\lambda^n V$]
  Similarly, one gets the identity
  \[
    s_{(1^m)}(x_1, \dots, x_n)
    = \sum_{1 \le j_1 < \cdots < j_m \le n} x_{j_1} \cdots x_{j_m}
    = e_m(x_1, \dots, x_m),
  \]
  the $m$th elementary symmetric function.
\end{example}

\begin{example}[Trace in $V^{\otimes N}$]
  Consider $x \otimes \sigma$, where
  $x = \diag(x_1, \dots, x_n)$ and
  $\sigma$ has $m_i$ cycles of
  length $i$. Then we have
  \[
    \tr_{V^{\otimes N}}(x \otimes \sigma)
    = \prod_i (x_1^i + \dots + x_n^i)^{m_i}.
  \]
  By Schur-Weyl duality, we have that
  \[
    \tr_{V^{\otimes N}}(x \otimes \sigma)
    = \sum_{\lambda} \chi_{\lambda}(\sigma) s_\lambda(x)
    = \prod_i (x_1^i + \dots + x_n^i)^{m_i}.
  \]
  Using the formula for the Schur
  polynomial, we get the identity
  \[
    \sum_{\lambda} \chi_\lambda(\sigma)
    \det(\{x_i^{\lambda_j + N - j}\})
    = \prod_{i < j} (x_i - x_j)
    \prod_i (x_1^i + \dots + x_n^i)^{m_i}.
  \]
\end{example}

\begin{theorem}[Frobenius character formula]
  $\chi_\lambda(\sigma)$ is the
  coefficient of $x_1^{\lambda_1 + N - 1} \cdots x_N^{\lambda_N}$
  in the polynomial
  \[\prod_{i < j} (x_i - x_j) \prod_i (x_1^i + \dots + x_n^i)^{m_i}.\]
\end{theorem}

\section{Howe Duality}

\begin{remark}
  Fix $V, W$ and consider
  $S^n(V \otimes W)$, which
  is a representation of
  $\GL(V) \otimes \GL(W)$.
\end{remark}

\begin{theorem}[Howe duality]
  We have a decomposition
  \[
    S^n(V \otimes W)
    = \bigoplus_{\lambda : |\lambda| = n} S^\lambda V \otimes S^\lambda W.
  \]
\end{theorem}

\begin{proof}
  We can write
  \[
    S^n(V \otimes W)
    = ((V \otimes W)^{\otimes n})^{S_n}
    = (V^{\otimes n} \otimes W^{\otimes n})^{S_n}.
  \]
  Using Schur-Weyl duality for each
  part, we get that
  \begin{align*}
    S^n(V \otimes W)
    &= \Bigg(\Big(\bigoplus_{\lambda : |\lambda| = n} S^\lambda V \otimes \pi_\lambda\Big)
    \otimes 
    \Big(
      \bigoplus_{\mu : |\mu| = n}
      S^\mu W \otimes \pi_{\mu}
    \Big)\Bigg)^{S_n} \\
    &= \bigoplus_{\lambda, \mu : |\lambda| = |\mu| = n}
    S^{\lambda} V \otimes
    S^{\mu} W \otimes (\pi_\lambda \otimes \pi_\mu)^{S_n}.
  \end{align*}
  Since $\pi_\lambda = \pi_{\lambda}^*$,
  by Schur's lemma we have
  $(\pi_\lambda \otimes \pi_\mu)^{S_n} = \C^{\delta_{\lambda, \mu}}$.
\end{proof}

\begin{corollary}[Cauchy identity]
  Let $x = (x_1, \dots, x_r)$ and
  $y = (y_1, \dots, y_s)$. Then
  \[
    \sum_\lambda
    s_\lambda(x) s_\lambda(y)
    z^{|\lambda|}
    = \prod_{i = 1}^r \prod_{j = 1}^s
    \frac{1}{1 - z x_i y_i}.
  \]
\end{corollary}

  \chapter{Jan.~21 --- Miniscule Weights}

\section{Miniscule Weights}

\begin{remark}
  Let $\g$ be a simple complex
  Lie algebra.
\end{remark}

\begin{definition}
  A dominant integral weight
  $\omega$ for $\g$ is called
  \emph{miniscule} if
  $\langle \omega, \beta \rangle \le 1$
  for every positive coroot $\beta$
  (equivalently, if
  $|\langle \omega, \alpha \rangle| \le 1$
  for any coroot $\beta$).
\end{definition}

\begin{example}
  Clearly $\omega = 0$
  is miniscule.
\end{example}

\begin{example}
  Let $\g = \mathfrak{sl}_n$ with
  fundamental weights
  $\{\omega_i\}_{i = 1}^{n - 1}$,\footnote{Recall a \emph{fundamental weight} is a weight $\omega_i$ such that $\langle \omega_i, \alpha_j^\vee \rangle = \delta_{i, j}$ for all simple coroots $\alpha_j^\vee$.}
  where
  \[
    \omega_i
    = (\underbrace{1, \dots, 1}_{i \text{ ones}}, 0, \dots, 0)
  \]
  Let $\alpha_{i, j} = \alpha_{i, j}^\vee = e_i - e_j$.
  Note that $\langle \omega_i, e_j - e_k \rangle = 0$
  when $j, k \le i$ or $j, k > i$, and
  $\langle \omega_i, e_j - e_k \rangle = 1$
  when $j \le i < k$.
  So all of the $\omega_i$
  are miniscule in this case.
\end{example}

\begin{lemma}
  Every nonzero miniscule weight
  is fundamental.
\end{lemma}

\begin{proof}
  Suppose $\omega$ is miniscule.
  Then there exists $i$ with
  $\langle \omega, \alpha_i^\vee \rangle = 1$.
  Moreover, there can only be one
  such $i$, since if there were
  many, then
  $\langle \omega, \theta^\vee \rangle \ge 2$,
  where $\theta^\vee$ is the longest
  coroot (i.e.
  if $\theta = \sum_{m_i > 0} m_i \alpha_i$ is
  the longest root, then
  $\theta^\vee = \sum_{m_i > 0} m_i \alpha_i^\vee$).
  So $\omega$ is necessarily
  fundamental.
\end{proof}

\begin{example}
  For $G_2$, $F_4$, and $F_8$, none
  of the fundamental weights are
  miniscule.
\end{example}

\begin{lemma}
  A fundamental weight $\omega_i$
  is miniscule if and only if
  $m_i = 1$ where
  $\theta^\vee = \sum_j m_j \alpha_j^\vee$.
\end{lemma}

\begin{proof}
  By the miniscule condition, we
  know $m_i \le 1$. If $m_i = 1$, then for
  any positive coroot $\beta = \sum n_j \alpha_j^\vee$
  we have $n_j \le m_j$, so
  $n_i \le 1$. Thus
  $\langle \omega_i, \beta \rangle = n_i \le 1$, so
  $\omega_i$ is miniscule.
\end{proof}

\begin{lemma}\label{lem:miniscule_zero}
  If $\omega \in Q$ with
  $|\langle \omega, \beta \rangle| \le 1$
  for all coroots $\beta$, then
  $\omega = 0$.
\end{lemma}

\begin{proof}
  Assume to the contrary that
  $\omega = \sum_i \alpha_i \ne 0$.
  We may assume that
  $\sum_i |m_i|$ is smallest possible.
  Then
  $0 < (\omega, \omega) = \sum_{i} m_i(\omega, \alpha_i)$,
  since the form is positive definite.
  Thus there exists $j$ such that
  $m_j$ and $\langle \omega, \alpha_j^\vee \rangle$
  have the same sign.
  By replacing $\omega$ with $-\omega$
  if necessary, we may assume both
  are positive. Then
  $\langle \omega, \alpha_j^\vee \rangle = 1$.
  Consider the reflection
  $s_j(\omega) = \omega - \alpha_j = \sum_i m_i' \alpha_i$.
  So $m_i' = m_j - 1$ and
  $m_i' = m_i$. But then
  $\sum_i |m_i'| = \sum_i |m_i| - 1 < \sum_i |m_i|$,
  contradicting the minimality
  of $\omega$.
\end{proof}

\begin{prop}
  The following conditions are
  equivalent:
  \begin{enumerate}
    \item $\omega$ is miniscule;
    \item all weights of
      $L_\omega$ belong to the
      Weyl orbit $W \omega$;
    \item if $\lambda$ is a dominant
      integral weight such that
      $\omega - \lambda \in Q_+$, then
      $\lambda = \omega$.
  \end{enumerate}
\end{prop}

\begin{proof}
  $(1 \Rightarrow 3)$
  If $\omega = 0$, then
  $-\lambda \in Q_+$, so
  $(\lambda, \rho) \le 0$
  where $\rho = \sum_{i = 1}^r \omega_i$,
  so $\lambda = 0$. Now let
  $\omega = \omega_i$ be miniscule.
  Then $\omega_i - \lambda = \sum_k m_k \alpha_k$
  with $m_k \ge 0$. If $m_k = 0$ for
  $k \ne i$,
  then the problem reduces to a lower
  rank Dynkin diagram. So we can
  assume $m_k > 0$ for every
  $k \ne i$. Let $\beta$ be a positive
  coroot, then
  \[
    \langle \omega_i - \lambda, \beta\rangle
    = \langle \omega_i, \beta \rangle
    - \langle \lambda, \beta \rangle
    \le \langle \omega_i, \beta \rangle
    \le 1.
  \]
  If $\alpha_i^\vee$ does not
  occur in $\beta$, then the above is
  $\le 0$. In particular,
  we have $\langle \omega_i - \lambda, \alpha_j^\vee \rangle \le 0$
  for $j \ne i$. If we also have
  $\langle \omega_i - \lambda, \alpha_i^\vee \rangle \le 0$,
  then $(\omega_i - \lambda, \omega_i - \lambda) \le 0$,
  so $\omega_i = \lambda$. Otherwise,
  $\langle \omega_i - \lambda, \alpha_i^\vee \rangle = 1$.
  Then $m_j > 0$ for every $j$,
  so $\langle \omega_i - \lambda, \theta^\vee \rangle \ge 1$, since
  $\theta^\vee$ is a dominant coweight.
  Then $\langle \lambda, \theta^\vee \rangle \le 0$,
  so we must have
  $\lambda = 0$ since
  $\theta^\vee$ contains all
  $\alpha_j^\vee$ with positive
  coefficients.
  But then $\omega_i \in Q$, which is
  impossible by
  Lemma \ref{lem:miniscule_zero}.

  $(3 \Rightarrow 2)$ If $\mu$ is
  any weight of $L_\omega$, then
  there exists $w \in W$ such that
  $\lambda = w \mu$ is dominant
  (since every orbit of $W$ intersects
  the dominant chamber at exactly
  $1$ point). Then $\omega - \lambda \in Q_+$,
  so $\lambda = \omega$, hence
  $\mu = w^{-1} \omega \in W \omega$.

  $(2 \Rightarrow 1)$ Suppose otherwise
  $\omega$ is not miniscule.
  Then $\langle \omega, \alpha^\vee \rangle > 1$
  for some positive coroot
  $\alpha^\vee$. Then
  \[
    2 (\omega, \alpha)
    > (\alpha, \alpha).
  \]
  Note that $\omega - \alpha$
  is a weight of $L_\omega$ (weight
  of $f_\alpha v_\omega$, where
  $v_\omega$ is a highest weight vector
  and $\{e_\alpha, f_\alpha, \alpha^\vee\}$
  is an $\mathfrak{sl}_2$-triple).
  But $\omega - \alpha$ is not
  $W$-conjugate to $\omega$, since
  \[
    (\omega - \alpha, \omega - \alpha)
    = (\omega, \omega)
    - 2(\omega, \alpha) + (\alpha, \alpha)
    < (\omega, \omega)
  \]
  but the pairing is $W$-invariant.
  Contradiction.
\end{proof}

\begin{corollary}
  If $\omega$ is miniscule, then
  $\chi_\omega = \sum_{\gamma \in W \omega} e^\gamma$.
\end{corollary}

\begin{prop}
  $\omega \in P_+$
  is miniscule if and only if
  the restriction of $L_\omega$
  to any root $\mathfrak{sl}_2$-subalgebra
  of $\g$ is the
  direct sum of $1$-dimensional
  and $2$-dimensional
  representations.
\end{prop}

\begin{proof}
  $(\Rightarrow)$
  Let $\omega$ be miniscule and
  $v \in L_\omega$ the highest
  weight vector (of weight $w \omega$)
  for $(\mathfrak{sl}_2)_\alpha$.
  Then
  \[
    h_\alpha v
    = \langle w \omega, \alpha^\vee \rangle v
    = \langle \omega, w^{-1} \alpha^\vee \rangle v.
  \]
  Then
  $h_\alpha v = 0$ or $h_\alpha v = v$,
  so the representation is
  $1$-dimensional or $2$-dimensional.

  $(\Leftarrow)$ Suppose
  $\omega$ is not miniscule.
  Then there exists $\alpha \in Q_+$ 
  with $\langle \omega, \alpha^\vee \rangle = m > 1$.
  Let $v_\omega$ be a highest
  weight vector, then
  $h_\alpha v_\omega = \langle \omega, \alpha^\vee \rangle v_\omega$, which
  leads to a higher-dimensional
  $\mathfrak{sl}_2$-representation.
\end{proof}

\begin{corollary}
  If $\omega$ is miniscule, then
  for every dominant integral
  weight $\lambda$ of $\g$, we have
  \[
    L_\omega \otimes L_\lambda
    = \bigoplus_{\gamma \in W \omega}
    L_{\lambda + \gamma}.
  \]
  (It is assumed that if
  $\lambda + \gamma$ is not dominant,
  then $L_{\lambda + \gamma} = 0$.)
\end{corollary}

\begin{proof}
  We know
  $\chi_\omega = \sum_{\mu \in W \omega} e^\mu$.
  Then we have
  \[
    \chi_{L_\omega \otimes L_\lambda}
    = \frac{\sum_{\mu \in W \omega} \sum_{w \in W} (-1)^{\ell(\omega)} e^{w(\lambda + \rho) + \mu}}{\Delta}
    = \frac{\sum_{\gamma \in W \omega} \sum_{w \in W} (-1)^{\ell(\omega)} e^{w(\lambda + \gamma + \rho)}}{\Delta}
  \]
  where $\Delta$ is the Weyl denominator.
  If $\lambda + \gamma \notin P_+$,
  then for some $\alpha_i^\vee$,
  we get
  $\langle \lambda + \gamma, \alpha_i^\vee \rangle < 0$. But we know
  $\langle \gamma, \alpha_i^\vee \rangle \ge -1$,
  so $\langle \lambda + \gamma, \alpha_i^\vee \rangle = -1$.
  Thus $\langle \lambda + \gamma + \rho, \alpha_i^\vee \rangle = 0$,
  so for any $w \gamma$,
  the term $w s_i \gamma$ comes
  with the opposite sign.
  So we get that
  \[
    \chi_{L_\omega \otimes L_\lambda}
    = \frac{\sum_{\gamma \in W \omega\, :\, \lambda + \gamma \in P_+}
    \sum_{w \in W} (-1)^{\ell(w)} e^{w(\lambda + \gamma + \rho)}}{\Delta}
    = \sum_{\gamma \in W \omega\, :\, \lambda + \gamma \in P_+}
    \chi_{\lambda + \gamma},
  \]
  which proves the desired result.
\end{proof}

\begin{example}
  For $\mathfrak{sl}_2$, we have
  $L_1 \otimes L_m = L_{m + 1} \oplus L_{m - 1}$,
  which leads to the formula
  \[
    L_m \otimes L_n
    = \bigoplus_{k = |m - n|}^{m + n}
    L_k
  \]
\end{example}

\begin{example}
  Let $V = V_{\omega_1}$ be the defining
  representation for $\GL_n$. Then
  \[
    L_{\omega_1} \otimes L_\lambda
    = \bigoplus_{\mu \in \lambda + \square} L_\mu,
  \]
  where $\lambda$ is a partition
  and $\lambda + \square$ denotes the
  set of partitions obtained by
  adding a single box to $\lambda$.
  For example, for
  $\lambda = (3, 3, 2, 1)$ we have
  \[
    L_{\omega_1} \otimes
    S^{(3, 3, 2, 1)} V
    = S^{(4, 3, 2, 1)} V
    \oplus S^{(3, 3, 3, 1)} V
    \oplus S^{(3, 3, 2, 2)} V
    \oplus S^{(3, 3, 2, 1, 1)} V.
  \]
  Similarly, for
  $\wedge^m V = L_{\omega_m}$
  (where $\omega_m = (1, \dots, 1, 0, \dots, 0)$
  with $m$ ones), we have
  \[
    L_{\omega_m} \otimes L_\lambda
    = \bigoplus_{\mu \in \lambda + m \square}
    L_\mu,
  \]
  where we are allowed to
  add $m$ boxes to $\lambda$
  in $\lambda + m \square$. For
  example,
  \[
    \wedge^2 V
    \otimes S^{(3, 1)} V
    = S^{(4, 2)} V
    \oplus S^{(4, 1, 1)} V
    \oplus S^{(3, 2, 1)} V
    \oplus S^{(3, 1, 1, 1)} V.
  \]
\end{example}

\end{document}
